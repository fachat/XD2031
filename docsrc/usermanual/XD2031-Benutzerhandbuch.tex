\documentclass[10pt,a4paper]{scrartcl}		% Ersatz für Klasse "article" 
\parskip 12pt
\parindent 0pt
\usepackage{makeidx}					% Stichwortverzeichnis
\makeindex
\makeglossary
\usepackage[ngerman]{babel}				% Umlaute "a "s etc.
\usepackage[utf8]{inputenc}				% Umlaute in Quelldatei UTF8-kodiert
\usepackage[T1]{fontenc}
\usepackage{footnpag}					% Nummerierung auf jeder Seite neu beginnen
\usepackage[hyphens]{url}
\usepackage{listings}
\lstloadlanguages{[Visual]Basic}
\lstset{language=C, basicstyle=\ttfamily\small, commentstyle=\itshape }
\usepackage{numprint}
\usepackage{booktabs}
\usepackage{amssymb}					% $\checkmark$
\usepackage[colorlinks,
pdfpagelabels,
pdfstartview = FitH,
bookmarksopen = true,
bookmarksnumbered = true,
linkcolor = black,
plainpages = false,
hypertexnames = false,
citecolor = black] {hyperref}
\usepackage{eurosym}
\usepackage{hyperref}
\usepackage{tikz}
\usetikzlibrary{trees}

% **************   SERIF   **************
%\usepackage{mathptmx} % Times Roman
%usepackage{fouriernc} % New Century Schoolbook
\usepackage{tgschola} % TEX Gyre Schola (based on New Century Schoolbook)
%\usepackage[scaled]{DejaVuSerif}

% **************   SANS SERIF   **************
%\usepackage[scaled=.90] {helvet}
\usepackage{tgadventor} % TEX Gyre Adventor (based on URW Gothic, an Avantgarde clone)
%\usepackage{tgheros} % TEX Gyre Heros (based on URW Nimbus Sans)
%\usepackage{arev} % Arev is a version of Bitstream Vera Sans, disigned for slide presentations

% **************   MONO   **************
%\usepackage{luximono}
\usepackage{beramono}

\newcommand{\beginsf}[1]{#1\begin{addmargin}[0.5cm]{0cm}}
\newcommand{\sfend}[1]{\end{addmargin}}

% -------------------------------------------------------------------------
% Variablen
%
\newcommand{\fwbinaries}{\url{http://xd2031.petsd.net/firmware.php}}
\newcommand{\osxbinaries}{\url{http://xd2031.petsd.net/serverosx.php}}
\newcommand{\srcdownloads}{\url{https://github.com/fachat/XD2031/tags}}
% -------------------------------------------------------------------------

\begin{document}



\tableofcontents
\clearpage

\section{Allgemeines}

\subsection{Verwendungszweck dieser Software}
\index{Verwendungszweck}
XD-2031 ist Software, die einen modernen Mikrocontroller für einen
historischen Commodore-Computer wie ein Diskettenlaufwerk erscheinen läßt.

Sie dient dem Datenaustausch zwischen alten und modernen Computern, der
ohne weiteres nicht mehr möglich ist. 
Moderne Computer können die Datenträger der alten nicht mehr lesen,
die alten aber auch nicht die der neuen und sie haben auch keine gemeinsame 
Schnittstelle, über die sie Daten untereinander austauschen könnten.

XD-2031 schließt diese Lücke mithilfe eines Adapters zwischen den
damals verwendeten Bus-Systemen und der aktuellen USB-Schnittstelle.

Damit kann man beispielsweise...
\begin{itemize}
\item alte Daten und Programme auf einem neuen PC sichern
\item aus dem Internet geladene Programme auf dem Commodore ausprobieren
\item mit modernen Werkzeugen Software für alte Rechner entwickeln
\end{itemize}

Dabei sind hier ausschließlich die 8-Bit Commodore-Computer gemeint, die über
einen IEEE-488-Bus \index{IEEE-488-Bus} oder seriellen
	\index{CBM-Bus} CBM-Bus\footnote
	{Diese proprietäre serielle Variante des IEEE-488-Busses wird
	oft auch als IEC-Bus \index{IEC-Bus|see{CBM-Bus}} bezeichnet.
	Da er allerdings \textit{nicht} durch die
	\textit{International Electrotechnical Commission} standardisiert
	wurde, verwendet diese Anleitung den Begriff CBM-Bus}
verfügen, wie etwa:
\begin{itemize}
\item C64, C128, VC20, +4
\item PET 2001\footnote{
	Nur mit BASIC 2, erkennbar an der Einschaltmeldung 
	\texttt{\#\#\# COMMODORE BASIC \#\#\#}}
, CBM 8032, CBM 610/710
\end{itemize}
Commodore Amiga und PC gehören nicht dazu, für sie existieren andere Lösungen.

\subsection{Über den Namen \glqq XD-2031\grqq}
Der Name steht in der Tradition der Namensgebung einer ganzen Familie
von Adapterkabeln, genannt die X-Kabel. Diese konnten zum 
Datenaustausch zwischen dem Diskettenlaufwerk \mbox{VC 1541} 
und PC verwendet werden, als diese noch über eine
Druckerschnittstelle nach Centronics-Standard verfügten.

Das X symbolisiert den Datenaustausch, D steht für \glqq drive\grqq\ 
und 2031 erinnert an das Diskettenlaufwerk CBM 2031.

\subsection{Unterstützte Adapter}
\index{Adapter}
\index{Geraet@Ger""at}
Derzeit läuft die Firmware auf folgenden Geräten:

\begin{tabular}[c]{l l}
\toprule 
Hardware & Homepage \\
\midrule
XS-1541	& \url{http://xs1541.t-winkler.net} \\
petSD	& \url{http://petsd.net} \\
\bottomrule
\end{tabular}

Für nähere Beschreibungen dieser Geräte sei auf deren Internetseiten
verwiesen.

\subsection*{Leistungsmerkmale nach Adapter}
\index{Leistungsmerkmale}
\index{Features}

\begin{tabular}[c]{l l l}
\toprule
			& XS-1541	& petSD			\\
\midrule
CBM-Bus			& $\checkmark$	& -- 			\\
IEEE-488-Bus		& $\checkmark$	& $\checkmark$		\\
Zugriff auf das Dateisystem
des Hosts		& $\checkmark$	& $\checkmark$		\\
ftp:// über Host	& $\checkmark$	& $\checkmark$		\\
http:// über Host	& $\checkmark$	& $\checkmark$		\\
Telnet über Host	& $\checkmark$	& $\checkmark$		\\
SD-Karte (FAT)		& --		& $\checkmark$		\\
\bottomrule
\end{tabular}

\subsection{Unterstützte Betriebssysteme}
\index{Betriebssysteme}
\begin{tabular}[c]{c c c}
\toprule
Linux & OS X & Windows \\
$\checkmark$ & $\checkmark$ & -- \\
\bottomrule
\end{tabular}

\section{Installation}
\subsection{USB-Treiber}
Unter Linux ist der Treiber schon seit einiger Zeit direkt im Kernel
enthalten, so dass keine separate Installation erfolgen muss.

Für OS X und Windows laden Sie sich bitte den Treiber von der
Herstellerseite \url{http://www.ftdichip.com/Drivers/VCP.htm} 
herunter und installieren ihn wie auf Ihrem System üblich.

\subsection{Aktualisierung der Firmware}
\index{Firmware}
Die XD-2031-Software, die auf einem Mikrocontroller-basiertem Adapter
läuft, wird auch als \textit{Firmware} bezeichnet.
Wenn Sie einen vorprogrammierten Adapter erhalten haben,
dürfen Sie direkt weiter bei \glqq\ref{serverinst} 
\nameref{serverinst}\grqq\ auf Seite \pageref{serverinst}  weiterlesen.

Wenn Sie ein petSD besitzen, lesen Sie bitte jetzt bei
\glqq\ref{petsdfwinst} \nameref{petsdfwinst}\grqq\ auf Seite
\pageref{petsdfwinst} weiter.

\subsubsection{Aktualisierung der Firmware für XS-1541}
Laden Sie zunächst von \fwbinaries{} die Firmware für das XS-1541
herunter und entpacken Sie wenn nötig Archivdateien wie etwa \glqq .zip\grqq . 
Halten Sie die entpackten Dateien in Ihrem Heimatverzeichnis bereit.
\subsubsection*{Linux}
\begin{enumerate}
\item Für die weiteren Arbeitsschritte werden Sie das Programm
\glqq AVRDUDE\grqq\ benötigen.  
Das Paket \glqq avrdude\grqq\ sollte in allen gängigen Distributionen
enthalten sein. Wie Sie dieses Paket installieren, hängt von Ihrer
verwendeten Distribution ab. Für Debian, Ubuntu und deren Verwandte:

\texttt{sudo apt-get install avrdude}

\item Schließen Sie das XS-1541 mittels USB-Kabel an Ihren Computer an.
\item Öffnen Sie ein Terminal, Sie sollten nun in Ihrem Heimatverzeichnis
sein, wo Ihre zuvor herunter geladenen Dateien liegen sollten.
\item Machen Sie die Script-Datei xs1541up.sh wie folgt ausführbar:

\texttt{chmod +x xs1541up.sh}
\item Wenn Sie die aktuelle Firmware verwenden wollen, tippen Sie
	folgenden Befehl ein, aber noch \textit{nicht} die 
	Eingabetaste:

	\texttt{./xs1541up.sh}

	\textbf{--- oder ---}

	Wenn Sie eine bestimmte Firmware-Version aufspielen möchten, geben
	Sie den Dateinamen an, drücken aber noch \textit{nicht}
	die Eingabetaste:

	\texttt{./xs1541up.sh XD2031-xs1541-bestimmte-Version.hex}
\item Drücken Sie am XS-1541 die Reset-Taste und lassen Sie sie wieder los.
\item Drücken Sie am PC binnen drei Sekunden die Eingabetaste um
	avrdude nun zu starten.
\end{enumerate}
\subsubsection*{OS X}

\begin{enumerate}
\item Installieren Sie das \textit{CrossPack for 
AVR\textsuperscript{\textregistered} Development} von 
\\\url{http://www.obdev.at/products/crosspack}
\item Schließen Sie das XS-1541 mittels USB-Kabel an Ihren Computer an.
\item Klicken Sie auf \glqq Programme\grqq\ $\to$ \glqq Dienstprogramme\grqq
$\to$ \glqq Terminal\grqq\ um ein Terminal zu öffnen.
\item Machen Sie die Script-Datei xs1541up.sh wie folgt ausführbar:

\texttt{chmod +x xs1541up.sh}
\item Tippen Sie den folgenden Befehl ein, aber noch \textit{nicht} die
Eingabetaste:

\texttt{sudo mv /dev/cu.usbserial}

\item Tippen Sie jetzt die Tab-Taste auf Ihrer Tastatur. Das Wort
\glqq usbserial\grqq\ sollte jetzt mit der Seriennummer Ihres Adapters
vervollständigt werden, wie z.B. usbserial-A10044Co.

Drücken Sie noch immer \textit{nicht} die Eingabetaste, sondern fügen
Sie die folgende Zeile an den Behl an:

\textit{<Leertaste>} \texttt{/dev/ttyUSB0}

Wenn der Befehl jetzt wie folgt aussieht (ausschließlich anderer
Seriennummer), drücken Sie jetzt die Eingabetaste:

\texttt{sudo mv /dev/cu.usbserial-A10044Co /dev/ttyUSB0}

Geben Sie dann Ihr Administrator-Passwort ein, um diesen Vorgang
zu ermöglichen.
\item Wenn Sie die aktuelle Firmware verwenden wollen, tippen Sie
	folgenden Befehl ein, aber noch \textit{nicht} die 
	Eingabetaste:

	\texttt{./xs1541up.sh}

	\textbf{--- oder ---}

	Wenn Sie eine bestimmte Firmware-Version aufspielen möchten, geben
	Sie den Dateinamen an, drücken aber noch \textit{nicht}
	die Eingabetaste:

	\texttt{./xs1541up.sh XD2031-xs1541-bestimmte-Version.hex}

\item Drücken Sie am XS-1541 die Reset-Taste und lassen Sie sie wieder los.
\item Drücken Sie am PC binnen drei Sekunden die Eingabetaste um
	avrdude nun zu starten.
\end{enumerate}

\subsubsection*{Wenn das Aktualisieren scheitert...}
Wenn es nicht geklappt hat, weil drei Sekunden doch recht kurz sind,
müssen Sie nicht alles noch einmal eintippen. Drücken Sie einfach
die $\uparrow$-Taste und der letzte Befehl sollte
wieder erscheinen. Drücken Sie dann Reset und jetzt schnell die Eingabetaste!

Sollte trotz allem die Fehlermeldung \glqq programmer is not responding\grqq\ 
erscheinen, wurde Ihr XS-1541 vermutlich \textit{nicht} mit einem Bootloader
versehen. In diesem Fall kann die Firmware nur unter Verwendung eines
AVR-ISP-Programmiergerätes aufgespielt werden.

\subsubsection{Aktualisierung der Firmware für petSD}
\label{petsdfwinst}
Haben Sie Ihr petSD nach Juli 2012 erworben, können Sie die Firmware
sehr einfach aktualisieren:
\begin{enumerate}
\item Laden Sie von \fwbinaries{} die Firmware für das \mbox{petSD}
herunter und entpacken Sie wenn nötig Archivdateien wie etwa \glqq .zip\grqq . 
\item Kopieren Sie die auf \glqq .bin\grqq\ endende Firmware-Datei
in das Hauptverzeichnis der am petSD verwendeten SD-Karte
\item Legen Sie die SD-Karte in das ausgeschaltete petSD ein
\item Schalten Sie das petSD ein
\end{enumerate}
Der Bootloader wird dann die neue Firmware finden und den Mikrocontroller
neu programmieren. Die Datei darf auch nach dem Programmiervorgang im 
Hauptverzeichnis liegen bleiben, da der Bootloader gleiche Firmwares
nur einmal programmiert.

Haben Sie dagegen Ihr petSD vor Juli 2012 erworben, kann es zu
Problemen kommen. Schreiben Sie in diesem Fall an nils.eilers@gmx.de, 
um weitere Informationen zu erhalten.

\subsection{Installation der Server-Software}
\label{serverinst}
\subsubsection{Installation der Server-Software unter Linux}
Die Installation aus dem letzten Stand des Programm-Quelltextes
ist mit ein paar Befehlen erledigt:
\lstset{language=,}
\begin{lstlisting}
git clone https://github.com/fachat/XD2031
cd XD2031/
make pcserver/fsser
sudo make install
\end{lstlisting}

Wenn Sie lieber eine bestimmte Release-Version installieren möchten, 
laden Sie von \srcdownloads{} den Quelltext herunter und führen im 
Unterverzeichnis des entpackten Quelltextes die folgenden Befehle aus:

\begin{lstlisting}
make pcserver/fsser
sudo make install
\end{lstlisting}

\subsubsection{Installation der Server-Software unter OS X}
Laden Sie von \osxbinaries{} den Server herunter und installieren ihn 
wie üblich.

\section{Erste Schritte}
Die meiste Arbeit ist nun geschafft. Bevor wir nun zum vergnüglichen
Teil kommen, sollen kurz drei Begriffe geklärt werden, die von nun
an in dieser Anleitung verwendet werden:
\begin{itemize}
\item Ihr moderner Computer wird fortan kurz \textit{Host}\footnote
	{Wenn diese Anleitung das Wort \glqq Host\grqq\ statt dem 
	verständlicherem \glqq PC\grqq\ verwendet, dann möchte sie betonen, 
	dass damit nicht unbedingt Computer nach dem IBM-PC-Standard 
	gemeint sind.}
genannt.
\item Ihr alter Commodore wird fortan der Einfachkeit halber als 
\textit{CBM} bezeichnet, ganz gleich, ob es sich um einen CBM / PET / C64 
oder anderen 8-Bitter handelt.
\item Ihr XS-1541 oder petSD wird als \textit{Adapter} bezeichnet.
\end{itemize}

\subsection{Adapter anschließen}
Auch wenn es hier erwähnt wird, sollten Sie vorerst auf das Anschließen
weiterer Diskettenlaufwerke verzichten, um Konflikte um Geräteadresse 8
auszuschließen.
\begin{enumerate}
\item Stellen Sie sicher, dass am Adapter \textit{keine(!)} Kabel angeschlossen sind.
\item Stellen Sie sicher, dass der CBM ausgeschaltet ist.
\item Wenn neben dem CBM noch weitere Hardware am IEEE-488-Bus oder CBM-Bus
angeschlossen ist, wie etwa weitere Diskettenlaufwerke oder Drucker,
müssen diese ebenfalls alle ausgeschaltet sein.
\item Der Host darf zu diesem Zeitpunkt bereits eingeschaltet sein. Falls er das noch nicht ist, schalten Sie ihn jetzt ein.
\item Verbinden Sie den Adapter mit dem CBM über ein IEEE-488- oder CBM-Bus-Kabel.
\item Verbinden Sie den Adapter mit dem Host mittels USB-Kabel.
\item Schalten Sie Peripherie wie Diskettenlaufwerke oder Drucker ein.
\item Wenn der Adapter über eine separate Stromzufuhr verfügt, schalten Sie diese jetzt ein.
\item Schalten Sie den CBM ein.
\end{enumerate}

Vielleicht staunen Sie über dieses Prozedere. Tatsache ist, das weder
IEEE-488-Bus noch CBM-Bus \textit{hot-pluggable} sind; das bedeutet:
Ein- oder Ausstecken nur im ausgeschalteten Zustand -- auch wenn man es in
der Praxis immer wieder anders sieht. Nichtbeachtung kann erforderlich
machen, dass Sie Ersatzteile für Ihren CBM besorgen müssen und Chips
aus- und einlöten müssen. Nebenbei bemerkt: der Bustreiber MC3446 ist
teuer geworden.

Falls Sie ein XS-1541 verwenden, und doch erst den CBM einschalten, bevor
das XS-1541 eingeschaltet ist, werden Sie ein Glimmen der Power-LED
beobachten können. Das rührt daher, dass die Spannung, die am Bus auch im
Ruhezustand anliegt, über die internen Sicherungsdioden des Mikrocontrollers
in das Stromnetz des Adapters fließt. Das ist ein Zustand, der
eindeutig außerhalb der \glqq absolute maximum ratings\grqq\ liegt,
die der Hersteller Atmel in seinem Datenblatt nennt. Atmel warnt,
dass dadurch die Zuverlässigkeit beeinträchtigt werden kann und bleibende
Schäden entstehen können. 

\subsection{XS-1541 ein- und ausschalten}
Natürlich wollen wir den Adapter jetzt eingeschaltet lassen, aber dies
ist die richtige Stelle, die Ein- und Ausschaltreihenfolge für das
XS-1541 zu besprechen. Beim petSD ist die Reihenfolge unerheblich.

\subsection*{XS-1541 einschalten}
\begin{enumerate}
\item Schalten Sie immer erst den Host ein. 
\item Stellen Sie sicher, dass das USB-Kabel am XS-1541 eingesteckt ist und 
die Power-LED hell leuchtet.
\item Wenn vorhanden, schalten Sie jetzt weitere Peripherie wie 
Diskettenlaufwerke oder Drucker ein.
\item Zuletzt schalten Sie den CBM ein, damit alle angeschlossenen
Geräte in den Grundzustand versetzt werden.
\end{enumerate}

\subsection*{XS-1541 ausschalten}
\begin{enumerate}
\item Schalten Sie den CBM und alle am Bus hängenden Peripheriegeräte
wie Diskettenlaufwerke und Drucker aus.
\item Ziehen Sie erst jetzt den USB-Stecker vom XS-1541 ab bzw.
schalten Sie erst jetzt den Host aus.
\end{enumerate}

\subsection{Server starten}

Die Konfiguration des Servers ist für diese Beispiele nicht erforderlich.
Sie wird später beschrieben.
Öffnen Sie ein Terminal. Unter OS X klicken Sie dazu auf \glqq Programme\grqq\  
$\to$ \glqq Dienstprogramme\grqq $\to$ \glqq Terminal\grqq\ .

Starten Sie dann den Server mit seiner Standard-Konfiguration, indem Sie
den folgenden Befehl eingeben und mit der Eingabetaste bestätigen:

\begin{verbatim}
fsser
\end{verbatim}


\subsection{Inhaltsverzeichnis laden}
Gegenüber dem CBM erscheint der Adapter wie ein klassisches Diskettenlaufwerk
unter einer einzigen Geräteadresse. Voreingestellt ist Geräteadresse 8.
So können Sie sich direkt wie früher das Inhaltsverzeichnis anzeigen
lassen:

\begin{verbatim}
LOAD"$",8
LIST
\end{verbatim}

Probieren Sie es aus. Sie sollten das Heimatverzeichnis Ihres Hosts
angezeigt bekommen.

\subsection{LOAD und SAVE}
Geben Sie ein kleines Programm ein:

\lstset{language=,} %[Visual]Basic}
\begin{lstlisting}
10 FOR I=1 TO 10
20 PRINT I,I*I
30 NEXT

RUN
\end{lstlisting}

Läuft es? Wenn Sie mit der Ausgabe zufrieden sind, speichern Sie Ihr Werk:

\texttt{SAVE"TEST",8}

Laden Sie dann erneut das Inhaltsverzeichnis:

\begin{verbatim}
LOAD"$",8
LIST
\end{verbatim}

Dort sollte nun neben den bereits bekannten Dateien auch Ihr
kleines Programm erscheinen. Nun hat das Inhaltsverzeichnis 
aber Ihr Programm im Speicher überschrieben, wovon Sie sich
gerne mit \texttt{LIST} oder \texttt{RUN} überzeugen können.
Holen Sie sich Ihr Programm zurück:

\begin{verbatim}
LOAD"TEST",8
RUN
\end{verbatim}


\subsection{Mehrere Laufwerke unter einer Geräteadresse}
\index{Laufwerke}
Unter einer Geräteadresse können mehrere Laufwerke gleichzeitig zur
Verfügung gestellt werden. 

Während das für Benutzer von Doppellaufwerken
wie CBM 8050 selbstverständlich ist, mag dieses Konzept für Benutzer
von Einzellaufwerken wie VC 1541 am C64 ungewohnt sein, weshalb es
hier kurz erklärt werden soll: vor Dateinamen kann die Nummer des gewünschten
Laufwerks durch Doppelpunkt getrennt angegeben werden. Geräte mit
mehreren Laufwerken wissen dann, auf welches Laufwerk der Zugriff
erfolgen soll.

Sehen wir uns doch einfach mal die dämliche Demo auf Laufwerk 1 an:

\begin{verbatim}
LOAD"$1:DUMB*",8
LIST

 0  "DUMB*           " XD2031
 1     "DUMBDEMO64"       PRG
 1     "DUMBDEMOPET"      PRG
 1     "DUMBDEMOVIC20"    PRG
 1  BLOCKS FREE
READY.
\end{verbatim}

Nicht gerade viel versprechend, aber sehen Sie selbst... 
Wählen Sie dazu den Dateinamen, der für Ihr System passt.

\begin{verbatim}
LOAD"DUMBDEMOPET",8
RUN
\end{verbatim}

Wieso wurde das Programm geladen, obwohl wir doch \texttt{LOAD"DUMBDEMOPET",8} 
geschrieben haben? Hätten wir nicht \texttt{LOAD"1:DUMBDEMOPET",8} schreiben müssen,
um von Laufwerk 1 zu laden?

Nun, \textit{ohne} Laufwerksnummer wird die \textit{zuletzt verwendete}
Laufwerksnummer verwendet. Und das war die von Laufwerk 1, von dem wir
das Inhaltsverzeichnis geladen haben.

Wenn Sie nun Ihr Testprogramm zurück erhalten wollen, können Sie
es sich mit \texttt{LOAD"0:TEST",8} wieder zurück holen.

XD-2031 erlaubt die Verwendung von bis zu 10 Laufwerken gleichzeitig,
die mit den vorgestellten Laufwerksziffern 0 bis 9 angesprochen
werden.

\subsection{BASIC 4.0 Disketten-Befehle}
Prinzipiell sind die zusätzlichen Disketten-Befehle des BASIC 4.0
nützlich und werden auch von der Firmware unterstützt. 
Eine gute Gelegenheit, das Hauptverzeichnis Ihres Hosts
wieder zu säubern:

\begin{verbatim}
SCRATCH"TEST",D0

ARE YOU SURE ?Y
01,FILES SCRATCHED,01,00
\end{verbatim}

Leider wird die Laufwerksangabe streng geprüft. Bei Angaben größer 1
wird ein \texttt{ILLEGAL QUANTITY ERROR} gemeldet. Dreifachlaufwerke
waren Commodore dann wohl doch nicht geheuer.

Immerhin sind ist bei der Angabe der Geräteadresse über ONU alle
sinnvollen Werte möglich. Mit \texttt{CATALOG ONU 20} würde das
Inhaltsverzeichnis eines Laufwerks an Geräteadresse 20 angezeigt werden.
Werte größer als 31 oder kleiner als drei werden verweigert. 

BASIC prüft hier zwar falsch, da die größtmögliche Geräteadresse nicht 31
sondern nur 30 ist, aber wenigstens stellt das keine Einschränkung dar.

\subsection{Geräteadresse ändern}
Das Ändern der Geräteadresse Ihres Adapters erfolgt durch Senden
eines Befehls an den Kommandokanal mit Sekundäradresse 15. Eine einfache
Möglichkeit, mit BASIC-Bordmitteln Befehle zu senden, ist das Anhängen
des Befehls an den OPEN-Befehl.

\begin{verbatim}
OPEN 1,8,15,"XU=30" : CLOSE 1
\end{verbatim}

Dies sendet dem Adapter an Geräteadresse 8 über den Kommandokanal 15
den Befehl, fortan die neue Geräteadresse 30 zu verwenden.

Wenn Sie BASIC 4 haben, probieren Sie ein \texttt{CATALOG ONU 30}, ansonsten:
\begin{verbatim}
LOAD"$",30
LIST
\end{verbatim}

Für die Assembler- oder C-Programmierer kann es einfacher sein,
die Laufwerksangabe binär statt ASCII-dezimal zu übertragen. Mit diesem
zugegeberweise BASIC-Beispiel hierfür stellen wir die Geräteadresse
wieder von 30 zurück auf 8:

\begin{verbatim}
OPEN 1,30,15
PRINT#1,"XU=";CHR$(8)
CLOSE 1
\end{verbatim}

Nach dem Ausschalten verwendet der Adapter wieder die voreingestellte
Geräteadresse 8. Folgende Software-Versionen werden die Möglichkeit 
bieten, den eingestellten Wert dauerhaft auch über das Ausschalten hinweg
beizubehalten.

\subsection{Laufwerken Inhalte zuweisen}
Bei klassischen Diskettenlaufwerken würden Sie eine Diskette 
einlegen. Hier muss dem Adapter mitgeteilt werden, \textit{wo} wo die Daten 
liegen, die als virtuelle Diskette erscheinen sollen.

Weil der Zugriff auf eine Festplatte aber grundsätzlich anders abläuft
als beispielsweise der Zugriff auf einen FTP-Server im Internet, 
muss man dem Adapter auch noch mitteilen \textit{wie} er darauf
zugreifen soll.

Für jeden Speicherort (Dateisystem/FTP/HTTP) gibt es daher einen 
\textit{Provider}. Auf Deutsch bedeutet das etwa \textit{Lieferant}
oder \textit{Leistungserbringer}. 

Hier ein Beispiel mit vier zugewiesenen Laufwerken:

\begin{tabular}[c]{c l l}
\toprule
Laufwerk & Provider & Ort \\
\midrule
0 & Dateisystem & sample/ \\
1 & Dateisystem & tools/ \\
2 & FTP & ftp.zimmers.net \\
3 & HTTP & www.gutenberg.org/ebooks/ \\
\bottomrule
\end{tabular}

\clearpage
\section{Dateien und Verzeichnisse}
\subsection{Regeln für Datei- und Verzeichnisnamen}
Als maximale Länge eines Datei- oder Verzeichnisnamens sollten
16 Zeichen nicht überschritten werden, damit alte Programme nicht
durch unerwartet lange Dateinamen Schwierigkeiten bekommen.

Generell sollten Sie Sonderzeichen in Dateinamen unbedingt 
vermeiden. Absolut verboten sind:
\begin{itemize}
\item Doppelpunkte :
\item Schrägstriche /
\item umgekehrte Schrägstriche \textbackslash
\item der Klammeraffe @
\item das Paragraphenzeichen $\S$
\item der Stern *
\item das Fragezeichen ? 
\item das Dollar-Zeichen \$
\end{itemize}
Leerzeichen können dagegen problemlos verwendet werden.
Wenn Sie einen sichtbaren Trenner verwenden möchten, sollten
Sie das Minus-Zeichen '-' bevorzugen, da der Unterstrich '\_' 
auf einem CBM schwierig einzugeben sein kann.

\subsection{Wildcards -- Dateimasken}
Wildcards funktionieren im CBM-Umfeld anders, als Sie es von DOS,
Windows oder unixoiden Betriebssystemen gewöhnt sein werden, weshalb 
Sie sich wenigstens den nächsten Abschnitt \glqq \nameref{starwc}\grqq\ 
ansehen sollten.

Aber zunächst sei erst einmal sei erklärt, was Wildcards oder Dateimasken 
überhaupt sind: nämlich
eine Art Joker für Dateinamen, die das Arbeiten mit mehreren Dateien
auf einmal sehr erleichtern.

Das Fragezeichen '?' steht dabei für ein beliebiges vorhandenes einzelnes 
Zeichen, das '*' für mehrere beliebige Zeichen oder nichts.

Angenommen, Sie hätten eine Liste von Dateien:

\begin{verbatim}
BUDGET1977
BUDGET1978
BUDGET1979
BUDGET1979.Q1
BUDGET1979.Q2
BUDGET1979.Q3
\end{verbatim}

\texttt{@\$:BUDGET197?} würde dann BUDGET1977, BUDGET1978 und BUDGET 1979
anzeigen, nicht aber BUDGET1979.Q1, BUDGET1979.Q2 und BUDGET1979.Q3.

\texttt{@\$:BUDGET1979?} würde gar keine Datei anzeigen, da es keine
Datei gibt, deren Name nach \glqq BUDGET1979\grqq\ noch genau ein
weiteres Zeichen enthält.

\texttt{@\$:BUDGET1979*} würde alle vier Dateien BUDGET1979 und 
BUDGET1979.Q1, BUDGET1979.Q2 und BUDGET1979.Q3 anzeigen, da bei
einem Stern nicht notwendigerweise ein Zeichen folgen muss.

\subsubsection*{Arbeitsweise der '*'-Wildcard}
\label{starwc}
Im Unterschied zu anderen Betriebssystem erfolgt nach '*' keine
weitere Auswertung. Wenn Sie also versuchen würden, die Daten
der ersten Quartale durch \texttt{*.Q1} zu kopieren, werden Sie
feststellen, dass statt dessen \textit{alle} Dateien kopiert werden.
Probieren Sie es aus, indem Sie sich mit \texttt{@\$:*.Q1} die
etroffenen Dateien anzeigen lassen.

\subsection{Verzeichnisse}
Um einzelne Verzeichnisnamen in Pfad-Angaben zu trennen, muss der
einfache Schrägstrich '/' verwendet werden.

Richtig: /Spiele/Strategie\\
Unzulässig: \textbackslash Spiele\textbackslash Strategie

Neben absoluten können auch relative Pfadangaben verwendet werden.
Was bedeutet das?

Nehmen wir an, Sie hätten Ihre Dateien 
in Unterverzeichnissen unter anderem wie folgt strukturiert:

\tikzstyle{every node}=[draw=black,thick,anchor=west]
\tikzstyle{selected}=[draw=red,fill=red!30]
\tikzstyle{optional}=[dashed,fill=gray!50]
\begin{tikzpicture}[%
  grow via three points={one child at (0.5,-0.7) and
    two children at (0.5,-0.7) and (0.5,-1.4)},
      edge from parent path={(\tikzparentnode.south) |- (\tikzchildnode.west)}]
\node {Hauptverzeichnis}
  child { node {Spiele} 
    child { node {Autorennen} }
    child { node {Ballern} }
    child { node [selected] {Strategie} }
  }
%  child [missing] {}
%  child [missing] {}
%  child [missing] {}
%  child { node {Texte} 
%    child { node {Arbeit} }
%    child { node {Briefe} }
%    child { node {Tagebuch} }
%  }
;
\end{tikzpicture}

Nehmen wir weiter an, sie befänden Sich im Unterverzeichnis Strategie
und wollten Sich nun in das Ballerspiel-Verzeichnis wechseln.

Absolute Pfadangaben beinhalten immer den kompletten Pfad vom
Hauptverzeichnis an. Mit \texttt{@CD:/SPIELE/BALLERN} würden Sie nun
das Verzeichnis wechseln.

Bei tiefer verschachtelten Verzeichnissen kann das eine Menge Tipp-Arbeit
bedeuten, weshalb sich relative Angaben anbieten. Zwei Punkte hintereinander
bezeichnen dabei immer das übergeordnete Verzeichnis, ohne dass
dieser Name ausdrücklich bekannt sein müsste. \texttt{@CD:../BALLERN}
bringt Sie dann direkt in das gewünschte Verzeichnis.

Mit \texttt{@CD:..} gelangen Sie immer eine Verzeichnisebene höher.

\texttt{@CD:/} bringt Sie in das Hauptverzeichnis.

\clearpage
\section{DOS-Befehle}
\subsection{Verwendete Notation}
\subsection*{Ersetzung durch tatsächlich passende Werte}
In spitzen Klammern und kursiver Schrift werden Werte angegeben, die 
Sie durch für Sie im Einzelfall tatsächlich
zutreffenden Werte ersetzen müssen. Beispiel:

\texttt{LOAD"}\textit{<Dateiname>}\texttt{",}\textit{<Geräteadresse>}

\texttt{LOAD"TEST",8}

\subsection*{Eine von mehreren Möglichkeiten wählen}
Soll aus mehreren Möglichkeiten genau eine Auswahl getroffen werden,
werden die Wahlmöglichkeiten in Klammern angegeben und durch einen
senkrechten Strich getrennt. Beispiel:

\mbox{\Big( Fisch \Big| \normalsize Fleisch \Big) \normalsize} 

Fleisch

\subsection*{Parameter angeben oder auslassen}
Wenn eine Angabe \textit{optional} ist, können Sie sie entweder
angeben oder weg lassen. Optionale Angaben stehen in eckigen Klammern.

\texttt{LOAD"} \Big[ Laufwerk-Nummer \texttt{:} \Big] \textit{<Dateiname>}
\texttt{",}\textit{<Geräteadresse>}

\texttt{LOAD"TEST",8} lädt die Datei TEST vom zuletzt 
verwendeten Laufwerk

\texttt{LOAD"2:TEST",8} lädt die Datei TEST von Laufwerk 2

\subsection*{Optionale Wiederholung}
Manche Befehle fordern mindestens einen Parameter, können aber beliebig
viele weitere\footnote{
	In der Praxis wird die Anzahl durch andere Umstände begrent, 
	wie etwa die Größe des Zwischenspeichers, der einen Befehl
	aufnehmen soll
} durch Komma getrennte Parameter annehmen. Die optionale Wiederholung
wird dann durch ein Komma gefolgt von drei Punkten dargestellt:

\mbox{\textit{<Name>} \Big[ \texttt{,} \ldots \ \Big]}

\begin{verbatim}
PETER
PETER,PAUL,MARY
\end{verbatim}

\subsection{I -- Initialize -- Laufwerk initialisieren}

\texttt{I} \Big[ \textit{<Laufwerk-Nummer>} \Big]

Dieser Befehl ist nur aus Kompatibilitätsgründen enthalten.
Seine einzige Wirkung ist das Rücksetzen der Statusmeldung.

Er stammt aus der Zeit, als Diskettenlaufwerke noch keine Schalter
eingebaut hatten, die einen Diskettenwechsel hätten erkennen können.
Mit diesem Befehl musste nach dem Einlegen einer Diskette dem
Laufwerk mitgeteilt werden, dass eine neue Diskette initialisiert
werden soll, also zur weiteren Benutzung vorbereitet werden soll.

Beispiele:
\begin{verbatim}
I
I0
\end{verbatim}

\subsection{R -- Rename -- Datei oder Verzeichnis umbenennen}
Benennt eine Datei oder ein Verzeichnis um. Beachten Sie die Reihenfolge:
der neue Dateiname wird zuerst genannt.

\texttt{R}\Big[ \textit{<Laufwerk-Nummer>} \Big] \texttt{:}
\textit{<Neuer Name>} \texttt{=} \textit{<Alter Name>}

Beispiele:
\begin{verbatim}
@R:REZEPTE PIZZEN=REZEPT PIZZA
@R0:WUENSCHE2013.TXT=WUENSCHE2012.TXT
\end{verbatim}

\subsection{S -- Scratch -- Datei(en) löschen}
\texttt{S} \Big[ \textit{<Laufwerk-Nummer>} \Big] \texttt{:}
\textit{<Dateimaske>} \Big[ \texttt{,} \ldots \  \Big]

Löscht die angegebenen Datei(en). Mehrere Dateinamen bzw. Suchmuster
können durch Komma getrennt angegeben werden.

Beispiele:

\texttt{@S:MIST} löscht die Datei MIST\\
\texttt{@S:STATISTIK198?.SEQ} löscht die Statistiken der Jahre 1980, 1981 
\ldots \ bis 1989 \\
\texttt{@S:MIST,UNSINN,ENTWURF,ENTWURF2} löscht alle genannten Dateien \\
\texttt{@S:*} löscht alle Dateien(!) des Unterverzeichnisses

\subsection{CD / CHDIR -- Change directory -- Unterverzeichnis wechseln}
\mbox{\Big( \texttt{CD} \Big| \texttt{CHDIR} \Big) 
\Big[ \textit{<Laufwerk-Nummer>} \Big]
\texttt{:} \textit{<Verzeichnis>}}

Wechselt in das angegebene Unterverzeichnis.

Beispiele:
\begin{verbatim}
@CD:SPIELE
@CD:..
@CD:TEXTE/HAUSAUFGABEN
\end{verbatim}

\subsection{MD / MKDIR -- Make directory}

\mbox{\Big( \texttt{MD} \Big| \texttt{MKDIR} \Big) 
\Big[ \textit{<Laufwerk-Nummer>} \Big]
\texttt{:} \textit{<Verzeichnis>}}

Erstellt das angegebene Unterverzeichnis. 

Bei Pfadangaben müssen
alle vorgenannten Verzeichnisse bereits existieren. Bei der
Angabe SOFTWARE/SPIELE/AUTORENNEN müssen also die Verzeichnisse
SOFTWARE/SPIELE bereits existieren, damit das Verzeichnis AUTORENNEN 
erstellt werden kann.

Beispiele:
\begin{verbatim}
@MD:SKIZZEN
@MD:TOPGAMES
\end{verbatim}



\subsection{RD / RMDIR -- Remove directory}

\mbox{\Big( \texttt{RD} \Big| \texttt{RMDIR} \Big) 
\Big[ \textit{<Laufwerk-Nummer>} \Big]
\texttt{:} \textit{<Verzeichnis>}}

Löscht das angegebene Unterverzeichnis, wenn es weder
Dateien noch Unterverzeichnisse enthält.

Beispiel:
\begin{verbatim}
@RM:EXPERIMENT2
\end{verbatim}




\subsection{XU -- Change unit -- Geräteadresse ändern}

\texttt{XU=}\Big( \textit{<Geräteadresse in ASCII Ziffern>} \Big|
\textit{<Geräteadresse binär>} \Big)

Soll die Einstellung dauerhaft auch nach dem Ausschalten erhalten 
bleiben, muss die Konfiguraton durch den Befehl \texttt{XW} gesichert werden.

Beispiele:

\begin{verbatim}
@XU=9
OPEN 1,8,15,"XU=9" : CLOSE 1
OPEN 1,8,15 : PRINT#1,"XU=";CHR$(9) : CLOSE 1
\end{verbatim}

\subsection{XD - Change last drive - Letztes Laufwerk setzen}

\texttt{XD=}\Big( \textit{<Laufwerk-Nummer in ASCII Ziffern>} \Big|
\textit{<Laufwerk-Nummer binär>} \Big)

Das zuletzt verwendete Laufwerk wird als \glqq Standardlaufwerk\grqq\  für
Zugriffe verwendet, bei denen im Dateinamen durch vorangestelltes
\textit{<Laufwerk-Nummer>}: nicht ausdrücklich ein bestimmtes
Laufwerk angegeben wird.  
Zusätzlich kann dieses Laufwerk auch mit dem XD Befehl gesetzt werden.

Soll die Einstellung dauerhaft auch nach dem Ausschalten erhalten 
bleiben, muss die Konfiguraton durch den Befehl \texttt{XW} gesichert werden.

Beispiele:

\begin{verbatim}
@XD=0
OPEN 1,8,15,"XD=0" : CLOSE 1
OPEN 1,8,15 : PRINT#1,"XD=";CHR$(0) : CLOSE 1
\end{verbatim}

\subsection{XI -- Init -- Einschaltkonfiguration wieder herstellen}

\texttt{XI}

Wenn der Adapter über einen nicht flüchtigen Speicher für die
Konfiguration besitzt, wird die dort gespeicherte Konfiguration beim
Einschalten des Adapters automatisch wieder hergestellt. 
Ist das nicht möglich, werden Standard-Werte eingesetzt.
Mit \texttt{XI} kann diese Einschaltkonfiguration jederzeit wieder
hergestellt werden.

Standardkonfiguration, falls keine gespeicherte Konfiguration verfügbar:

\begin{tabular}[c]{l l l}
\toprule 
Einstellung & Befehl & Wert\\
\midrule
Geräteadresse		& XU= & 8 \\
Letztes Laufwerk	& XD= & 0 \\
\bottomrule
\end{tabular}

\subsection{XW -- Write configuration -- Konfiguration schreiben}

\texttt{XW}

Speichert die aktuelle Konfiguration in einem nicht flüchtigem Speicher.
Beim nächsten Einschalten des Adapters werden die gespeicherten Werte
automatisch wieder hergestellt. Eine Übersicht der gespeicherten
Einstellungen ist in der Beschreibung des \texttt{XI}-Befehls enthalten.

\subsection{XRESET - Adapter zurücksetzen}

\texttt{XRESET}

Dieser Befehl ist die Software-Entsprechung zum Druck auf die Reset-Taste
des Adapters. Alle nicht gespeicherten Daten und Einstellungen gehen
dabei verloren.

Wenn der Adapter über einen Bootloader zur Aktualisierung der Firmware
verfügt, wird dieser vor dem Starten der Firmware aktiv. \texttt{XRESET}
kann daher dazu verwendet werden, den Aktualisierungsvorgang anzustoßen.

\section{Anhang}
\subsection{Verwendete Lizenzen}
\clearpage
\begin{center}

\mbox{\bf\huge GNU GENERAL PUBLIC LICENSE}

{\bf\large Version 2, June 1991}
\bigskip

{\parindent 0in

Copyright \copyright\ 1989, 1991 Free Software Foundation, Inc.

\bigskip

51 Franklin Street, Fifth Floor, Boston, MA  02110-1301, USA

\bigskip

Everyone is permitted to copy and distribute verbatim copies
of this license document, but changing it is not allowed.
}

{\bf\large Preamble}
\end{center}

The licenses for most software are designed to take away your freedom to
share and change it.  By contrast, the GNU General Public License is
intended to guarantee your freedom to share and change free software---to
make sure the software is free for all its users.  This General Public
License applies to most of the Free Software Foundation's software and to
any other program whose authors commit to using it.  (Some other Free
Software Foundation software is covered by the GNU Library General Public
License instead.)  You can apply it to your programs, too.

When we speak of free software, we are referring to freedom, not price.
Our General Public Licenses are designed to make sure that you have the
freedom to distribute copies of free software (and charge for this service
if you wish), that you receive source code or can get it if you want it,
that you can change the software or use pieces of it in new free programs;
and that you know you can do these things.

To protect your rights, we need to make restrictions that forbid anyone to
deny you these rights or to ask you to surrender the rights.  These
restrictions translate to certain responsibilities for you if you
distribute copies of the software, or if you modify it.

For example, if you distribute copies of such a program, whether gratis or
for a fee, you must give the recipients all the rights that you have.  You
must make sure that they, too, receive or can get the source code.  And
you must show them these terms so they know their rights.

We protect your rights with two steps: (1) copyright the software, and (2)
offer you this license which gives you legal permission to copy,
distribute and/or modify the software.

Also, for each author's protection and ours, we want to make certain that
everyone understands that there is no warranty for this free software.  If
the software is modified by someone else and passed on, we want its
recipients to know that what they have is not the original, so that any
problems introduced by others will not reflect on the original authors'
reputations.

Finally, any free program is threatened constantly by software patents.
We wish to avoid the danger that redistributors of a free program will
individually obtain patent licenses, in effect making the program
proprietary.  To prevent this, we have made it clear that any patent must
be licensed for everyone's free use or not licensed at all.

The precise terms and conditions for copying, distribution and
modification follow.

\begin{center}
{\Large \sc Terms and Conditions For Copying, Distribution and
  Modification}
\end{center}


%\renewcommand{\theenumi}{\alpha{enumi}}
\begin{enumerate}

\addtocounter{enumi}{-1}

\item 

This License applies to any program or other work which contains a notice
placed by the copyright holder saying it may be distributed under the
terms of this General Public License.  The ``Program'', below, refers to
any such program or work, and a ``work based on the Program'' means either
the Program or any derivative work under copyright law: that is to say, a
work containing the Program or a portion of it, either verbatim or with
modifications and/or translated into another language.  (Hereinafter,
translation is included without limitation in the term ``modification''.)
Each licensee is addressed as ``you''.

Activities other than copying, distribution and modification are not
covered by this License; they are outside its scope.  The act of
running the Program is not restricted, and the output from the Program
is covered only if its contents constitute a work based on the
Program (independent of having been made by running the Program).
Whether that is true depends on what the Program does.

\item You may copy and distribute verbatim copies of the Program's source
  code as you receive it, in any medium, provided that you conspicuously
  and appropriately publish on each copy an appropriate copyright notice
  and disclaimer of warranty; keep intact all the notices that refer to
  this License and to the absence of any warranty; and give any other
  recipients of the Program a copy of this License along with the Program.

You may charge a fee for the physical act of transferring a copy, and you
may at your option offer warranty protection in exchange for a fee.

\item

You may modify your copy or copies of the Program or any portion
of it, thus forming a work based on the Program, and copy and
distribute such modifications or work under the terms of Section 1
above, provided that you also meet all of these conditions:

\begin{enumerate}

\item 

You must cause the modified files to carry prominent notices stating that
you changed the files and the date of any change.

\item

You must cause any work that you distribute or publish, that in
whole or in part contains or is derived from the Program or any
part thereof, to be licensed as a whole at no charge to all third
parties under the terms of this License.

\item
If the modified program normally reads commands interactively
when run, you must cause it, when started running for such
interactive use in the most ordinary way, to print or display an
announcement including an appropriate copyright notice and a
notice that there is no warranty (or else, saying that you provide
a warranty) and that users may redistribute the program under
these conditions, and telling the user how to view a copy of this
License.  (Exception: if the Program itself is interactive but
does not normally print such an announcement, your work based on
the Program is not required to print an announcement.)

\end{enumerate}


These requirements apply to the modified work as a whole.  If
identifiable sections of that work are not derived from the Program,
and can be reasonably considered independent and separate works in
themselves, then this License, and its terms, do not apply to those
sections when you distribute them as separate works.  But when you
distribute the same sections as part of a whole which is a work based
on the Program, the distribution of the whole must be on the terms of
this License, whose permissions for other licensees extend to the
entire whole, and thus to each and every part regardless of who wrote it.

Thus, it is not the intent of this section to claim rights or contest
your rights to work written entirely by you; rather, the intent is to
exercise the right to control the distribution of derivative or
collective works based on the Program.

In addition, mere aggregation of another work not based on the Program
with the Program (or with a work based on the Program) on a volume of
a storage or distribution medium does not bring the other work under
the scope of this License.

\item
You may copy and distribute the Program (or a work based on it,
under Section 2) in object code or executable form under the terms of
Sections 1 and 2 above provided that you also do one of the following:

\begin{enumerate}

\item

Accompany it with the complete corresponding machine-readable
source code, which must be distributed under the terms of Sections
1 and 2 above on a medium customarily used for software interchange; or,

\item

Accompany it with a written offer, valid for at least three
years, to give any third party, for a charge no more than your
cost of physically performing source distribution, a complete
machine-readable copy of the corresponding source code, to be
distributed under the terms of Sections 1 and 2 above on a medium
customarily used for software interchange; or,

\item

Accompany it with the information you received as to the offer
to distribute corresponding source code.  (This alternative is
allowed only for noncommercial distribution and only if you
received the program in object code or executable form with such
an offer, in accord with Subsection b above.)

\end{enumerate}


The source code for a work means the preferred form of the work for
making modifications to it.  For an executable work, complete source
code means all the source code for all modules it contains, plus any
associated interface definition files, plus the scripts used to
control compilation and installation of the executable.  However, as a
special exception, the source code distributed need not include
anything that is normally distributed (in either source or binary
form) with the major components (compiler, kernel, and so on) of the
operating system on which the executable runs, unless that component
itself accompanies the executable.

If distribution of executable or object code is made by offering
access to copy from a designated place, then offering equivalent
access to copy the source code from the same place counts as
distribution of the source code, even though third parties are not
compelled to copy the source along with the object code.

\item
You may not copy, modify, sublicense, or distribute the Program
except as expressly provided under this License.  Any attempt
otherwise to copy, modify, sublicense or distribute the Program is
void, and will automatically terminate your rights under this License.
However, parties who have received copies, or rights, from you under
this License will not have their licenses terminated so long as such
parties remain in full compliance.

\item
You are not required to accept this License, since you have not
signed it.  However, nothing else grants you permission to modify or
distribute the Program or its derivative works.  These actions are
prohibited by law if you do not accept this License.  Therefore, by
modifying or distributing the Program (or any work based on the
Program), you indicate your acceptance of this License to do so, and
all its terms and conditions for copying, distributing or modifying
the Program or works based on it.

\item
Each time you redistribute the Program (or any work based on the
Program), the recipient automatically receives a license from the
original licensor to copy, distribute or modify the Program subject to
these terms and conditions.  You may not impose any further
restrictions on the recipients' exercise of the rights granted herein.
You are not responsible for enforcing compliance by third parties to
this License.

\item
If, as a consequence of a court judgment or allegation of patent
infringement or for any other reason (not limited to patent issues),
conditions are imposed on you (whether by court order, agreement or
otherwise) that contradict the conditions of this License, they do not
excuse you from the conditions of this License.  If you cannot
distribute so as to satisfy simultaneously your obligations under this
License and any other pertinent obligations, then as a consequence you
may not distribute the Program at all.  For example, if a patent
license would not permit royalty-free redistribution of the Program by
all those who receive copies directly or indirectly through you, then
the only way you could satisfy both it and this License would be to
refrain entirely from distribution of the Program.

If any portion of this section is held invalid or unenforceable under
any particular circumstance, the balance of the section is intended to
apply and the section as a whole is intended to apply in other
circumstances.

It is not the purpose of this section to induce you to infringe any
patents or other property right claims or to contest validity of any
such claims; this section has the sole purpose of protecting the
integrity of the free software distribution system, which is
implemented by public license practices.  Many people have made
generous contributions to the wide range of software distributed
through that system in reliance on consistent application of that
system; it is up to the author/donor to decide if he or she is willing
to distribute software through any other system and a licensee cannot
impose that choice.

This section is intended to make thoroughly clear what is believed to
be a consequence of the rest of this License.

\item
If the distribution and/or use of the Program is restricted in
certain countries either by patents or by copyrighted interfaces, the
original copyright holder who places the Program under this License
may add an explicit geographical distribution limitation excluding
those countries, so that distribution is permitted only in or among
countries not thus excluded.  In such case, this License incorporates
the limitation as if written in the body of this License.

\item
The Free Software Foundation may publish revised and/or new versions
of the General Public License from time to time.  Such new versions will
be similar in spirit to the present version, but may differ in detail to
address new problems or concerns.

Each version is given a distinguishing version number.  If the Program
specifies a version number of this License which applies to it and ``any
later version'', you have the option of following the terms and conditions
either of that version or of any later version published by the Free
Software Foundation.  If the Program does not specify a version number of
this License, you may choose any version ever published by the Free Software
Foundation.

\item
If you wish to incorporate parts of the Program into other free
programs whose distribution conditions are different, write to the author
to ask for permission.  For software which is copyrighted by the Free
Software Foundation, write to the Free Software Foundation; we sometimes
make exceptions for this.  Our decision will be guided by the two goals
of preserving the free status of all derivatives of our free software and
of promoting the sharing and reuse of software generally.

\begin{center}
{\Large\sc
No Warranty
}
\end{center}

\item
{\sc Because the program is licensed free of charge, there is no warranty
for the program, to the extent permitted by applicable law.  Except when
otherwise stated in writing the copyright holders and/or other parties
provide the program ``as is'' without warranty of any kind, either expressed
or implied, including, but not limited to, the implied warranties of
merchantability and fitness for a particular purpose.  The entire risk as
to the quality and performance of the program is with you.  Should the
program prove defective, you assume the cost of all necessary servicing,
repair or correction.}

\item
{\sc In no event unless required by applicable law or agreed to in writing
will any copyright holder, or any other party who may modify and/or
redistribute the program as permitted above, be liable to you for damages,
including any general, special, incidental or consequential damages arising
out of the use or inability to use the program (including but not limited
to loss of data or data being rendered inaccurate or losses sustained by
you or third parties or a failure of the program to operate with any other
programs), even if such holder or other party has been advised of the
possibility of such damages.}

\end{enumerate}


\begin{center}
{\Large\sc End of Terms and Conditions}
\end{center}


\pagebreak[2]

\section*{Appendix: How to Apply These Terms to Your New Programs}

If you develop a new program, and you want it to be of the greatest
possible use to the public, the best way to achieve this is to make it
free software which everyone can redistribute and change under these
terms.

  To do so, attach the following notices to the program.  It is safest to
  attach them to the start of each source file to most effectively convey
  the exclusion of warranty; and each file should have at least the
  ``copyright'' line and a pointer to where the full notice is found.

\begin{quote}
one line to give the program's name and a brief idea of what it does. \\
Copyright (C) yyyy  name of author \\

This program is free software; you can redistribute it and/or modify
it under the terms of the GNU General Public License as published by
the Free Software Foundation; either version 2 of the License, or
(at your option) any later version.

This program is distributed in the hope that it will be useful,
but WITHOUT ANY WARRANTY; without even the implied warranty of
MERCHANTABILITY or FITNESS FOR A PARTICULAR PURPOSE.  See the
GNU General Public License for more details.

You should have received a copy of the GNU General Public License
along with this program; if not, write to the Free Software
Foundation, Inc., 51 Franklin Street, Fifth Floor, Boston, MA  02110-1301, USA.
\end{quote}

Also add information on how to contact you by electronic and paper mail.

If the program is interactive, make it output a short notice like this
when it starts in an interactive mode:

\begin{quote}
Gnomovision version 69, Copyright (C) yyyy  name of author \\
Gnomovision comes with ABSOLUTELY NO WARRANTY; for details type `show w'. \\
This is free software, and you are welcome to redistribute it
under certain conditions; type `show c' for details.
\end{quote}


The hypothetical commands {\tt show w} and {\tt show c} should show the
appropriate parts of the General Public License.  Of course, the commands
you use may be called something other than {\tt show w} and {\tt show c};
they could even be mouse-clicks or menu items---whatever suits your
program.

You should also get your employer (if you work as a programmer) or your
school, if any, to sign a ``copyright disclaimer'' for the program, if
necessary.  Here is a sample; alter the names:

\begin{quote}
Yoyodyne, Inc., hereby disclaims all copyright interest in the program \\
`Gnomovision' (which makes passes at compilers) written by James Hacker. \\

signature of Ty Coon, 1 April 1989 \\
Ty Coon, President of Vice
\end{quote}


This General Public License does not permit incorporating your program
into proprietary programs.  If your program is a subroutine library, you
may consider it more useful to permit linking proprietary applications
with the library.  If this is what you want to do, use the GNU Library
General Public License instead of this License.

\clearpage


% Erzeugt falschen Link im PDF
% \addcontentsline{toc}{subsection}{Stichwortverzeichnis}

\renewcommand{\indexname}{Stichwortverzeichnis}
\printindex

\end{document}




