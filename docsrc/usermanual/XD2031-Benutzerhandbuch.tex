\documentclass[10pt,a4paper]{scrartcl}		% Ersatz für Klasse "article" 
\parskip 12pt
\parindent 0pt
\usepackage{makeidx}					% Stichwortverzeichnis
\makeindex
% \makeglossary
\usepackage[utf8]{inputenc}				% Umlaute in Quelldatei UTF8-kodiert
\usepackage[T1]{fontenc}
\usepackage{footnpag}					% Nummerierung auf jeder Seite neu beginnen
\usepackage[hyphens]{url}
\usepackage{listings}
\lstloadlanguages{[Visual]Basic}
\lstset{language=C, basicstyle=\ttfamily\small, commentstyle=\itshape }
\usepackage{numprint}
\usepackage{booktabs}
\usepackage{amssymb}					% $\checkmark$
\usepackage[colorlinks,
pdfpagelabels,
pdfstartview = FitH,
bookmarksopen = true,
bookmarksnumbered = true,
linkcolor = black,
plainpages = false,
hypertexnames = false,
citecolor = black] {hyperref}
\usepackage{eurosym}
\usepackage{hyperref}
\usepackage{tikz}
\usetikzlibrary{trees}
\usepackage{verbatim}
\usepackage{graphicx}
\usepackage[ngerman]{babel}				% Umlaute "a "s etc.
\RequirePackage[ngerman=ngerman-x-latest]{hyphsubst}

\hyphenation{zwi-schen}


% **************   SERIF   **************
%\usepackage{mathptmx} % Times Roman
%usepackage{fouriernc} % New Century Schoolbook
\usepackage{tgschola} % TEX Gyre Schola (based on New Century Schoolbook)
%\usepackage[scaled]{DejaVuSerif}

% **************   SANS SERIF   **************
%\usepackage[scaled=.90] {helvet}
\usepackage{tgadventor} % TEX Gyre Adventor (based on URW Gothic, an Avantgarde clone)
%\usepackage{tgheros} % TEX Gyre Heros (based on URW Nimbus Sans)
%\usepackage{arev} % Arev is a version of Bitstream Vera Sans, disigned for slide presentations

% **************   MONO   **************
%\usepackage{luximono}
\usepackage{beramono}

\newcommand{\beginsf}[1]{#1\begin{addmargin}[0.5cm]{0cm}}
\newcommand{\sfend}[1]{\end{addmargin}}

% -------------------------------------------------------------------------
% Global variables for LaTeX documents
% -------------------------------------------------------------------------

% Internet links
\newcommand{\fwbinaries}{\url{http://xd2031.petsd.net/firmware.php}}
\newcommand{\osxbinaries}{\url{http://xd2031.petsd.net/serverosx.php}}
\newcommand{\winbinaries}{\url{http://xd2031.petsd.net/serverwin.php}}
\newcommand{\srcdownloads}{\url{http://xd2031.petsd.net/sources.php}}


\begin{document}

\title{Benutzerhandbuch}
\subtitle{XD-2031 Serial line filesystem server}
\author{}
\maketitle\thispagestyle{empty}

\clearpage

\tableofcontents
\clearpage

\section{Allgemeines}

\subsection{Verwendungszweck dieser Software}
\index{Verwendungszweck}
XD-2031 ist Software, die einen modernen Mikrocontroller für einen
historischen Commodore-Computer wie ein Diskettenlaufwerk erscheinen läßt.

Sie dient dem Datenaustausch zwischen alten und modernen Computern, der
ohne weiteres nicht mehr möglich ist. 
Moderne Computer können die Datenträger der alten nicht mehr lesen,
die alten aber auch nicht die der neuen und sie haben auch keine gemeinsame 
Schnittstelle, über die sie Daten austauschen könnten.

XD-2031 schließt diese Lücke mithilfe eines Adapters zwischen den
damals verwendeten Bus-Systemen und der aktuellen USB-Schnittstelle.

Damit kann man beispielsweise...
\begin{itemize}
\item alte Daten und Programme auf einem neuen PC sichern
\item aus dem Internet geladene Programme auf dem Commodore ausprobieren
\item mit modernen Werkzeugen Software für alte Rechner entwickeln
\end{itemize}

Dabei sind hier ausschließlich die 8-Bit Commodore-Computer gemeint, die 
über einen parallelen IEEE-488-Bus \index{IEEE-488-Bus} oder seriellen
	\index{IEC-Bus} \index{CBM-Bus} IEC-Bus\footnote
	{Diese proprietäre serielle Variante des IEEE-488-Busses wird
	in deutschsprachiger Literatur auch als CBM-Bus bezeichnet,
	da er \textit{nicht} durch die
	\textit{International Electrotechnical Commission} standardisiert
	wurde.}
verfügen:
\begin{itemize}
\item C64, C128, VC20, +4
\item CBM 3032, 
PET 2001\footnote{
	Nur mit BASIC 2, erkennbar an der Einschaltmeldung 
	\texttt{\#\#\# COMMODORE BASIC \#\#\#}}
\item CBM 4032, CBM 8032, CBM 8096, CBM 8296 
\item CBM 610, CBM 710
\end{itemize}
Commodore Amiga und PC gehören nicht dazu, allerdings existieren 
für diese Rechner andere Lösungen, wie etwa der HxC Floppy Emulator.

\subsection{Über \glqq XD-2031\grqq}
Der Name steht in der Tradition der Namensgebung einer ganzen Familie
von Adapterkabeln, genannt die X-Kabel. Diese konnten zum 
Datenaustausch zwischen dem Diskettenlaufwerk \mbox{VC 1541} 
und PC verwendet werden, als diese noch über eine
Druckerschnittstelle nach Centronics-Standard verfügten.

Das X symbolisiert den Datenaustausch, D steht für \glqq drive\grqq\ 
und 2031 erinnert an das Diskettenlaufwerk CBM 2031.

Auch wenn die Namensgebung die Emulation eines bestimmten Laufwerktyps
vermuten lassen könnte, ist dies nicht der Fall. Es wird auch keine
in den Original-Laufwerken enthaltene 6502 CPU emuliert.
Tatsächlich werden auf DOS-Ebene Dateianfragen beantwortet. Die
Verwendung von Disk-Images (D64, D80 ...) 
ist allerdings möglich.

\subsection{Unterstützte Adapter}
\index{Adapter}
\index{Geraet@Ger""at}
Derzeit läuft die Firmware auf folgenden Geräten:

\begin{tabular}[c]{l l}
\toprule 
Hardware & Homepage \\
\midrule
XS-1541	& \url{http://xs1541.t-winkler.net} \\
Raspberry Pi mit 3.1541 & -- \\
petSD	& \url{http://petsd.net} \\
\bottomrule
\end{tabular}

Für nähere Beschreibungen dieser Geräte sei auf deren Internetseiten
verwiesen.

\subsection*{Leistungsmerkmale nach Adapter}
\index{Leistungsmerkmale}
\index{Features}

\begin{tabular}[c]{l l l l}
\toprule
			& XS-1541	& 3.1541       & petSD		\\
\midrule
IEC-Bus			& $\checkmark$	& $\checkmark$ &  --		\\
IEEE-488-Bus		& $\checkmark$	& --	       & $\checkmark$	\\
Zugriff auf XD-2031 Serverdienste
			& $\checkmark$	& $\checkmark$ & $\checkmark$	\\
SD-Karte (FAT)		& --		& --	       & $\checkmark$	\\
\bottomrule
\end{tabular}

\subsection{Unterstützte Betriebssysteme}
\index{Betriebssysteme}
\begin{tabular}[c]{c c c}
\toprule
Linux & OS X & Windows \\
$\checkmark$ & $\checkmark$ & -- \\
\bottomrule
\end{tabular}

\section{Installation}
\subsection{USB-Treiber}
Unter Linux ist der Treiber schon seit einiger Zeit direkt im Kernel
enthalten, so dass keine separate Installation erfolgen muss.

Für OS X und Windows laden Sie sich bitte den Treiber von der
Herstellerseite \url{http://www.ftdichip.com/Drivers/VCP.htm} 
herunter und installieren ihn wie auf Ihrem System üblich.

\subsection{Raspberry Pi unter Raspbian mit 3.1541}
Das 3.1541 wird direkt an den UART des Raspberry Pi angeschlossen.
Die gängigen Linux-Systeme verwenden diesen UART allerdings von
Hause aus als Konsole um unter anderem Kernel-Meldungen auszugeben. 
Da diese Meldungen die Datenübertragung zwischen Raspberry Pi
und dem Coprozessor stören würden, muss der UART erst frei
gegeben werden. Damit der R-Pi den Coprozessor zurücksetzen kann,
wird die Wiring Pi Bibliothek benötigt.

\subsubsection*{Kernelmeldungen nicht mehr auf UART ausgeben}
Die Datei \texttt{/boot/cmdline.txt} enthält die Kernelparameter, die ihm
beim Start übergeben werden.  
Erstellen Sie eine Sicherungskopie, nur für den Fall der Fälle:

\begin{verbatim}
sudo cp /boot/cmdline.txt /boot/cmdline.txt~
\end{verbatim}

Bearbeiten Sie nun mit Root-Rechten die Datei \texttt{/boot/cmdline.txt}:

\begin{verbatim}
sudo vi /boot/cmdline.txt
\end{verbatim}

Entfernen Sie dort alle Parameter, die den seriellen Port \texttt{ttyAMA0}
ansprechen, was am Beispiel hier die folgenden waren:

\begin{verbatim}
console=ttyAMA0,115200 kgdboc=ttyAMA0,115200
\end{verbatim}

Die verbleibende Zeile lautete hier:
\begin{verbatim}
dwc_otg.lpm_enable=0 console=tty1 root=/dev/mmcblk0p2 
rootfstype=ext4 elevator=deadline rootwait
\end{verbatim}

Speichern Sie Ihre Änderungen.

\subsubsection*{Kein Login mehr über UART}
Bearbeiten Sie nun wieder unter
Root-Rechten die Datei \texttt{/etc/inittab}:

\begin{verbatim}
sudo vi /etc/inittab
\end{verbatim}

Durchsuchen Sie die Datei nach \glqq{}ttyAMA0 \grqq{}, bis Sie 
etwa folgendes haben:

\begin{verbatim}
T0:23:respawn:/sbin/getty -L ttyAMA0 115200 vt100
\end{verbatim}

Dieser Eintrag erlaubt es, sich über die serielle Schnittstelle
\texttt{/dev/ttyAMA0} mittels eines Terminals anzumelden.
Damit der Datenverkehr vom 3.1541 nicht als Benutzername und Passwort
mißverstanden wird, wird dieser Eintrag durch das Voranstellen einer Raute 
auskommentiert:

\begin{verbatim}
#T0:23:respawn:/sbin/getty -L ttyAMA0 115200 vt100
\end{verbatim}

Speichern Sie Ihre Änderungen und starten Sie den Raspberry Pi neu.

\begin{verbatim}
sudo shutdown -r now
\end{verbatim}

\subsubsection*{Installation von Wiring Pi}
Falls bei Ihrer Linux-Distribution \textit{Wiring Pi}\footnote{\url{http://wiringpi.com/}} 
noch nicht enthalten ist, muss es noch installiert werden:

\begin{verbatim}
sudo apt-get update
sudo apt-get upgrade
sudo apt-get install git-core
git clone git://git.drogon.net/wiringPi
cd wiringPi
./build
\end{verbatim}

Die weitere Installation erfolgt wie bei jedem anderen Debian/Ubuntu-artigem Linux.

\subsection{Installation der Server-Software}
\label{serverinst}
\subsubsection{Installation der Server-Software unter Linux}
Für Linux stehen leider noch keine Binärpakete zur Verfügung,
so dass Sie hier aus dem Quelltext selbst compilieren müssen.
Das ist allerdings leichter, als es klingt.

Alle für die weitere Installation benötigten Pakete sollten
sich durch einen Klick auf diesen 
\href{apt://build-essential,gcc-avr,avr-libc,avrdude,avrdude-doc,libcurl4-gnutls-dev,git}{apturl-Link} installieren lassen.

Ansonsten kopieren Sie folgende Anweisung in die Kommandozeile, 
achten Sie dabei darauf, dass alles in \textit{einer} Zeile steht:

\begin{verbatim}
sudo apt-get install build-essential gcc-avr avr-libc 
avrdude avrdude-doc libcurl4-gnutls-dev git
\end{verbatim}

Sie haben nun die Wahl, ob Sie den Entwicklerstand oder eine
bestimme Release-Version verwenden möchten. Ein Entwicklerstand
enthält die aktuelle Fassung und enthält die neuesten Features,
Fehlerbehebungen und möglicherweise auch neue Fehler.

\subsubsection*{Installation aus dem Entwicklerstand}

\begin{verbatim}
git clone https://github.com/fachat/XD2031
cd XD2031/pcserver
make
sudo make install
\end{verbatim}

Fertig. Wenn Sie später eine neuere Fassung verwenden möchten, müssen Sie
nicht wieder alles herunter laden, sondern können folgende Befehle
anwenden:

\begin{verbatim}
cd XD2031/pcserver
git pull
make
sudo make install
\end{verbatim}

\subsubsection*{Installation einer Release-Version}
Laden Sie von \srcdownloads{}  den Quelltext her"-unter und
entpacken Sie ihn. Wechseln Sie in das Quelltextverzeichnis
und führen die folgenden Befehle aus:

\begin{lstlisting}
cd XD2031/pcserver
make
sudo make install
\end{lstlisting}

\subsubsection*{Automatischer Start des Servers}
Wenn Sie den Server beim Hochfahren des Rechners automatisch mit starten 
soll\footnote{Bislang nur für Raspbian auf dem Raspberry Pi getestet!},
führen Sie noch den folgenden Befehl aus:

\begin{verbatim}
sudo make daemon
\end{verbatim}

Um den Server manuell zu starten oder zu beenden:

\begin{verbatim}
sudo service fsser start
sudo service fsser stop
\end{verbatim}

Meldungen des Servers werden in \texttt{/var/log/fsser} 
abgelegt. Sie können wie folgt live mitgelesen werden:

\begin{verbatim}
tail -f /var/log/fsser
\end{verbatim}

\subsubsection{Installation der Server-Software unter OS X}
Laden Sie von \osxbinaries{} den Server her"-unter und installieren ihn 
wie üblich.

\subsection{Aktualisierung der Firmware}
\index{Firmware}
Die XD-2031-Software, die auf einem Mikrocontroller-basiertem Adapter
läuft, wird auch als \textit{Firmware} bezeichnet.

Sicherheitshalber sollten Sie nach dem Installieren einer neuen
Server-Version auch die Firmware aktualisieren, damit beide richtig
zusammen arbeiten können.

Anweisungen für Raspberry Pi / 3.1541 finden Sie unter
bei \glqq\ref{fwinst31541} \nameref{fwinst31541}\grqq\ auf Seite
\pageref{fwinst31541}.

Wenn Sie ein petSD besitzen, lesen Sie bitte jetzt bei
\glqq\ref{petsdfwinst} \nameref{petsdfwinst}\grqq\ auf Seite
\pageref{petsdfwinst} weiter.

\textbf{Tipp:} Um direkt zur gewünschten Stelle zu gelangen, können 
Sie auch innerhalb der PDF-Datei die Kapitelnamen oder Seitenzahlen 
anklicken.

\subsubsection{Aktualisierung der Firmware für XS-1541}
Laden Sie zunächst von \fwbinaries{} die Firm"-ware für das XS-1541
herunter und entpacken Sie wenn nötig Archivdateien wie 
etwa \glqq .zip\grqq . 

\subsubsection*{Linux}
\begin{enumerate}
\item Schließen Sie das XS-1541 mittels USB-Kabel an Ihren Computer an.
\item Öffnen Sie ein Terminal.
\item Wechseln Sie in das Verzeichnis, das die entpackte Firmware enthält.
\item Machen Sie die Script-Datei xs1541up.sh wie folgt ausführbar:

\texttt{chmod +x xs1541up.sh}
\item Für die aktuelle Firmware \texttt{XD2031-xs1541.hex}, tippen Sie
	den folgenden Befehl ein, aber noch \textit{nicht} die 
	Eingabetaste:

	\texttt{./xs1541up.sh}

\item Drücken Sie am XS-1541 die Reset-Taste und lassen Sie sie wieder los.
\item Drücken Sie am PC binnen drei Sekunden(!) die Eingabetaste
\end{enumerate}

\textbf{Tipp:} Wenn Sie statt der vorgegebenen Datei 
\texttt{XD2031-xs1541.hex}
eine andere Firmware aufspielen möchten, können Sie
dem Script auch einen Dateinamen übergeben:

\begin{verbatim}
./xs1541up.sh Beliebige-xs1541-Firmware.hex
\end{verbatim}

\subsubsection*{OS X}

\begin{enumerate}
\item Installieren Sie das \textit{CrossPack for 
AVR\textsuperscript{\textregistered} Development} von 
\\\url{http://www.obdev.at/products/crosspack}
\item Schließen Sie das XS-1541 mittels USB-Kabel an Ihren Computer an.
\item Klicken Sie auf \glqq Programme\grqq\ $\to$ \glqq Dienstprogramme\grqq
$\to$ \glqq Terminal\grqq\ um ein Terminal zu öffnen.
\item Wechseln Sie in das Verzeichnis, das die entpackte Firmware enthält.
\item Machen Sie die Script-Datei xs1541up.sh wie folgt ausführbar:

\texttt{chmod +x xs1541up.sh}
\item Tippen Sie den folgenden Befehl ein, aber noch \textit{nicht} die
Eingabetaste:

\texttt{sudo mv /dev/cu.usbserial}

\item Tippen Sie jetzt die Tab-Taste auf Ihrer Tastatur. Das Wort
\glqq usbserial\grqq\ sollte jetzt mit der Seriennummer Ihres Adapters
vervollständigt werden, wie z.B. usbserial-A10044Co.

Drücken Sie noch immer \textit{nicht} die Eingabetaste, sondern fügen
Sie die folgende Zeile an den Behl an:

\textit{<Leertaste>} \texttt{/dev/ttyUSB0}

Wenn der Befehl jetzt wie folgt aussieht (ausschließlich anderer
Seriennummer), drücken Sie jetzt die Eingabetaste:

\texttt{sudo mv /dev/cu.usbserial-A10044Co /dev/ttyUSB0}

Geben Sie dann Ihr Administrator-Passwort ein, um diesen Vorgang
zu ermöglichen.
\item Für die aktuelle Firmware 
   \texttt{XD2031-xs1541.hex} tippen Sie
	folgenden Befehl ein, aber noch \textit{nicht} die 
	Eingabetaste:

	\texttt{./xs1541up.sh}

\item Drücken Sie am XS-1541 die Reset-Taste und lassen Sie sie wieder los.
\item Drücken Sie am PC binnen drei Sekunden(!) die Eingabetaste
\end{enumerate}

\textbf{Tipp:} Wenn Sie statt der vorgegebenen Datei 
\texttt{XD2031-xs1541.hex}
eine andere Firmware aufspielen möchten, können Sie
dem Script auch einen Dateinamen übergeben:

\begin{verbatim}
./xs1541up.sh Beliebige-xs1541-Firmware.hex
\end{verbatim}

\subsubsection*{Wenn das Aktualisieren scheitert...}
Wenn es nicht geklappt hat, weil drei Sekunden doch recht kurz sind,
müssen Sie nicht alles noch einmal eintippen. Drücken Sie einfach
die $\uparrow$-Taste und der letzte Befehl sollte
wieder erscheinen. Drücken Sie dann Reset und jetzt schnell die Eingabetaste!

Sollte trotz allem die Fehlermeldung \glqq programmer is not responding\grqq\ 
erscheinen, wurde Ihr XS-1541 vermutlich \textit{nicht} mit einem Bootloader
versehen. In diesem Fall kann die Firmware nur unter Verwendung eines
AVR-ISP-Programmierers aufgespielt werden.



\subsubsection{Aktualisierung der Firmware für Raspberry Pi / 3.1541}
\label{fwinst31541}
Laden Sie zunächst von \fwbinaries{} die Firm\-ware für das 3.1541
herunter und entpacken Sie wenn nötig Archivdateien wie etwa \glqq .zip\grqq . 
Das 3.1541 verwendet die selbe Firmware die das XS-1541, sie werden daher
keine separaten Dateien finden. Allerdings ist der Update-Vorgang etwas
leichter:

\begin{enumerate}
\item Wechseln Sie in das Verzeichnis, das die entpackte Firmware enthält.
\item Machen Sie die Script-Datei 3.1541.sh wie folgt ausführbar:

\texttt{chmod +x 3.1541up.sh}
\item Um die aktuelle Firmware \texttt{XD2031-xs1541.hex} aufzuspielen:

	\texttt{./3.1541up.sh}
\end{enumerate}

\textbf{Tipp:} Wenn Sie statt der vorgegebenen Datei 
\texttt{XD2031-xs1541.hex}
eine andere Firmware aufspielen möchten, können Sie
dem Script auch einen Dateinamen übergeben:

\begin{verbatim}
./3.1541up.sh Beliebige-xs1541-Firmware.hex
\end{verbatim}

\subsubsection{Aktualisierung der Firmware für petSD}
\label{petsdfwinst}
Zwar wurden die 2011er petSD mit einem Bootloader zum einfachen
Aktualisieren der Firmware über die SD-Karte versehen, es hat
sich jedoch gezeigt, dass die verwendeten Fuses-Einstellungen
des AVR Microcontrollers nicht mit der 
\mbox{XD-2031} Firmware kompatibel sind.

Leider können Bootloader nur die Firmware aktualisieren, 
aber keine Änderungen an den Fuses vornehmen. 
Die einzige Möglichkeit, die XD-2031 Firmware
auf ein petSD aufzuspielen, besteht daher in der Verwendung eines
AVR-ISP-Programmiergerätes. 

Ein solches Programmiergerät kann je nach Hersteller 
für einen Betrag von ca. 20,-- bis 30,-- \euro{} erworben werden. 
Das Atmel AVR ISP mkII ist eine Empfehlung wert und wird in allen
Scripten als Standard verwendet.

\subsubsection*{Download der Firmware und Einstellen des
Programmier-Gerätes}
\begin{enumerate}
\item Unter OS X installieren Sie bitte erst das 
\textit{CrossPack for AVR\textsuperscript{\textregistered} Development} von 
\url{http://www.obdev.at/products/crosspack}

\item Laden Sie von \fwbinaries{} die Firmware herunter und entpacken Sie 
die Datei

\item Öffnen Sie ein Terminal. Unter OS X klicken Sie dazu auf 
\glqq Programme\grqq\ $\to$ \glqq Dienstprogramme\grqq
$\to$ \glqq Terminal\grqq{}

\item Wechseln Sie in das Verzeichnis, das die entpackte Firmware enthält
\end{enumerate}

Wenn Sie ein anderes Programmiergerät als das Atmel AVR ISP mkII
verwenden, müssen die zunächst die für Ihr Programmiergerät passenden
Werte eintragen. Lesen Sie dazu die Anleitung Ihres Programmiergerätes oder 
die Online-Hilfe mit \texttt{man avrdude}

\begin{enumerate}
\setcounter{enumi}{4}
\item Bearbeiten Sie wenn nötig die Datei \texttt{programmer.cfg} 
und ändern Sie die beiden Parameter für Typ und Anschluss des 
Programmiergerätes in pas\-sende Werte:
\begin{verbatim}
PROGRAMMER=avrispmkii
CONNECTED_TO=usb
\end{verbatim}
\end{enumerate}

\subsubsection*{Anpassen der Fuses}
Dieser Schritt ist nur einmalig notwendig und sollte bei
weiteren Firmware-Updates ausgelassen werden.
\begin{enumerate}
\item Machen Sie das Script \texttt{petSD\_fix\_fuses.sh} ausführbar:
\begin{verbatim}
chmod +x petSD_fix_fuses.sh
\end{verbatim}

\item Starten Sie das Script, um die Fuses für XD-2031 zu ändern:
\begin{verbatim}
./petSD_fix_fuses.sh
\end{verbatim}

\end{enumerate}

\subsubsection*{Aktualisieren der Firmware mit AVR ISP Programmierer}

\begin{enumerate}
\item Machen Sie das Script \texttt{petSDisp.sh} ausführbar:
\begin{verbatim}
chmod +x petSDisp.sh
\end{verbatim}

\item Um die aktuelle Firmware \texttt{XD2031-petSD.hex} aufzuspielen:

\texttt{./petSDisp.sh}
\end{enumerate}

\textbf{Tipp:} Wenn Sie statt der vorgegebenen Datei 
\texttt{XD2031-petSD.hex}
eine andere Firmware aufspielen möchten, können Sie
dem Script auch einen Dateinamen übergeben:

\begin{verbatim}
./petSDisp.sh Beliebige-petSD-Firmware.hex
\end{verbatim}

\begin{comment}
Haben Sie Ihr petSD nach Juli 2012 erworben, können Sie die Firmware
sehr einfach aktualisieren:
\begin{enumerate}
\item Laden Sie von \fwbinaries{} die Firmware für das \mbox{petSD}
herunter und entpacken Sie wenn nötig Archivdateien wie etwa \glqq .zip\grqq . 
\item Kopieren Sie die auf \glqq .bin\grqq\ endende Firmware-Datei
in das Hauptverzeichnis der am petSD verwendeten SD-Karte
\item Legen Sie die SD-Karte in das ausgeschaltete petSD ein
\item Schalten Sie das petSD ein
\end{enumerate}
Der Bootloader wird dann die neue Firmware finden und den Mikrocontroller
neu programmieren. Die Datei darf auch nach dem Programmiervorgang im 
Hauptverzeichnis liegen bleiben, da der Bootloader gleiche Firmwares
nur einmal programmiert.

Haben Sie dagegen Ihr petSD vor Juli 2012 erworben, kann es zu
Problemen kommen. Schreiben Sie in diesem Fall an nils.eilers@gmx.de, 
um weitere Informationen zu erhalten.
\end{comment}

\clearpage
\section{Erste Schritte}
Die meiste Arbeit ist nun geschafft. Bevor wir nun zum vergnüglichen
Teil kommen, sollen kurz drei Begriffe geklärt werden, die von nun
an in dieser Anleitung verwendet werden:
\begin{itemize}
\item Ihr moderner Computer wird fortan der Einfachkeit halber 
	\glqq{}PC\grqq{} genannt.
\item Ihr alter Commodore wird als 
\glqq{}CBM\grqq{} bezeichnet, ganz gleich, ob es sich um einen CBM / PET / C64 
oder anderen 8-Bitter handelt.
\item Ihr XS-1541, 3.1541 oder petSD wird 
als \glqq{}Adapter\grqq{} bezeichnet.
\end{itemize}

\subsection{Adapter anschließen}
Auch wenn es hier erwähnt wird, sollten Sie vorerst auf das Anschließen
weiterer Diskettenlaufwerke verzichten, um Konflikte um Geräteadresse 8
auszuschließen.
\begin{enumerate}
\item Stellen Sie sicher, dass am Adapter \textit{keine(!)} Kabel angeschlossen sind.
\item Stellen Sie sicher, dass der CBM ausgeschaltet ist.
\item Wenn neben dem CBM noch weitere Hardware am IEEE-488-Bus oder IEC-Bus
angeschlossen ist, wie etwa weitere Diskettenlaufwerke oder Drucker,
müssen diese ebenfalls alle ausgeschaltet sein.
\item Der PC darf zu diesem Zeitpunkt bereits eingeschaltet sein. Falls er das noch nicht ist, schalten Sie ihn jetzt ein.
\item Verbinden Sie den Adapter mit dem CBM über ein IEEE-488- oder IEC-Bus-Kabel.
\item Verbinden Sie den Adapter mit dem PC mittels USB-Kabel.
\item Schalten Sie Peripherie wie Diskettenlaufwerke oder Drucker ein.
\item Wenn der Adapter über eine separate Stromzufuhr verfügt, schalten Sie diese jetzt ein.
\item Schalten Sie den CBM ein.
\end{enumerate}

Vielleicht staunen Sie über dieses Prozedere. Tatsache ist, das weder
IEEE-488-Bus noch IEC-Bus \textit{hot-pluggable} sind; das bedeutet:
Ein- oder Ausstecken nur im ausgeschalteten Zustand -- auch wenn man es in
der Praxis immer wieder anders sieht. Nichtbeachtung kann erforderlich
machen, dass Sie Ersatzteile für Ihren CBM besorgen müssen und Chips
aus- und einlöten müssen. Nebenbei bemerkt: der Bustreiber MC3446 ist
teuer geworden.

Falls Sie ein XS-1541 verwenden, und doch erst den CBM einschalten, bevor
das XS-1541 eingeschaltet ist, werden Sie ein Glimmen der Power-LED
beobachten können. Das rührt daher, dass die Spannung, die am Bus auch im
Ruhezustand anliegt, über die internen Sicherungsdioden des Mikrocontrollers
in das Stromnetz des Adapters fließt. Das ist ein Zustand, der
eindeutig außerhalb der \glqq absolute maximum ratings\grqq\ liegt,
die der Hersteller Atmel in seinem Datenblatt nennt. Atmel warnt,
dass dadurch die Zuverlässigkeit beeinträchtigt werden kann und bleibende
Schäden entstehen können. 

\subsection{XS-1541 ein- und ausschalten}
Natürlich wollen wir den Adapter jetzt eingeschaltet lassen, aber dies
ist die richtige Stelle, die Ein- und Ausschaltreihenfolge für das
XS-1541 zu besprechen. Beim petSD ist die Reihenfolge unerheblich.

\subsection*{XS-1541 einschalten}
\begin{enumerate}
\item Schalten Sie immer erst den PC ein. 
\item Stellen Sie sicher, dass das USB-Kabel am XS-1541 eingesteckt ist und 
die Power-LED hell leuchtet.
\item Wenn vorhanden, schalten Sie jetzt weitere Peripherie wie 
Diskettenlaufwerke oder Drucker ein.
\item Zuletzt schalten Sie den CBM ein, damit alle angeschlossenen
Geräte in den Grundzustand versetzt werden.
\end{enumerate}

\subsection*{XS-1541 ausschalten}
\begin{enumerate}
\item Schalten Sie den CBM und alle am Bus hängenden Peripheriegeräte
wie Diskettenlaufwerke und Drucker aus.
\item Ziehen Sie erst jetzt den USB-Stecker vom XS-1541 ab bzw.
schalten Sie erst jetzt den PC aus.
\end{enumerate}

\subsection{Server starten}

Die Konfiguration des Servers ist für diese Beispiele nicht erforderlich.
Sie wird später beschrieben.
Öffnen Sie ein Terminal. Unter OS X klicken Sie dazu auf \glqq Programme\grqq\  
$\to$ \glqq Dienstprogramme\grqq $\to$ \glqq Terminal\grqq\ .

Starten Sie dann den Server mit seiner Standard-Konfiguration, indem Sie
den folgenden Befehl eingeben und mit der Eingabetaste bestätigen:

\begin{verbatim}
fsser
\end{verbatim}


\subsection{Inhaltsverzeichnis laden}
Gegenüber dem CBM erscheint der Adapter wie ein klassisches Diskettenlaufwerk
unter einer einzigen Geräteadresse. Voreingestellt ist Geräteadresse 8.
So können Sie sich direkt wie früher das Inhaltsverzeichnis anzeigen
lassen:

\begin{verbatim}
LOAD"$",8
LIST
\end{verbatim}

Probieren Sie es aus. Sie sollten das Heimatverzeichnis Ihres PCs
angezeigt bekommen.

\subsection{LOAD und SAVE}
Geben Sie ein kleines Programm ein:

\lstset{language=,} %[Visual]Basic}
\begin{lstlisting}
10 FOR I=1 TO 10
20 PRINT I,I*I
30 NEXT

RUN
\end{lstlisting}

Läuft es? Wenn Sie mit der Ausgabe zufrieden sind, speichern Sie Ihr Werk:

\texttt{SAVE"TEST",8}

Laden Sie dann erneut das Inhaltsverzeichnis:

\begin{verbatim}
LOAD"$",8
LIST
\end{verbatim}

Dort sollte nun neben den bereits bekannten Dateien auch Ihr
kleines Programm erscheinen. Nun hat das Inhaltsverzeichnis 
aber Ihr Programm im Speicher überschrieben, wovon Sie sich
gerne mit \texttt{LIST} oder \texttt{RUN} überzeugen können.
Holen Sie sich Ihr Programm zurück:

\begin{verbatim}
LOAD"TEST",8
RUN
\end{verbatim}


\subsection{Mehrere Laufwerke unter einer Geräteadresse}
\index{Laufwerke}
Unter einer Geräteadresse können mehrere Laufwerke gleichzeitig zur
Verfügung gestellt werden. 

Während das für Benutzer von Doppellaufwerken
wie CBM 8050 selbstverständlich ist, mag dieses Konzept für Benutzer
von Einzellaufwerken wie VC 1541 am C64 ungewohnt sein, weshalb es
hier kurz erklärt werden soll: vor Dateinamen kann die Nummer des gewünschten
Laufwerks durch Doppelpunkt getrennt angegeben werden. Geräte mit
mehreren Laufwerken wissen dann, auf welches Laufwerk der Zugriff
erfolgen soll.

Sehen wir uns doch einfach mal die dämliche Demo auf Laufwerk 1 an:

\begin{verbatim}
LOAD"$1:DUMB*",8
LIST

 1 "DUMB*           "XD2031
 1     "DUMBDEMO64"      PRG
 1     "DUMBDEMOPET"     PRG
 1     "DUMBDEMOVIC20"   PRG
 63999 BLOCKS FREE.
READY.
\end{verbatim}

Nicht gerade viel versprechend, aber sehen Sie selbst... 
Wählen Sie dazu den Dateinamen, der für Ihr System passt.

\begin{verbatim}
LOAD"DUMBDEMOPET",8
RUN
\end{verbatim}

Wieso wurde das Programm geladen, obwohl wir doch \texttt{LOAD"DUMBDEMOPET",8} 
geschrieben haben? Hätten wir nicht \texttt{LOAD"1:DUMBDEMOPET",8} schreiben müssen,
um von Laufwerk 1 zu laden?

Nun, \textit{ohne} Laufwerksnummer wird die \textit{zuletzt verwendete}
Laufwerksnummer verwendet. Und das war die von Laufwerk 1, von dem wir
das Inhaltsverzeichnis geladen haben.

Wenn Sie nun Ihr Testprogramm zurück erhalten wollen, können Sie
es sich mit \texttt{LOAD"0:TEST",8} wieder zurück holen.

XD-2031 erlaubt die Verwendung von bis zu 10 Laufwerken gleichzeitig,
die mit den vorgestellten Laufwerksziffern 0 bis 9 angesprochen
werden.

\subsection{BASIC 4.0 Disketten-Befehle}
Prinzipiell sind die zusätzlichen Disketten-Befehle des BASIC 4.0
nützlich und werden auch von der Firmware unterstützt. 
Eine gute Gelegenheit, das Hauptverzeichnis Ihres PCs
wieder zu säubern:

\begin{verbatim}
SCRATCH"TEST",D0

ARE YOU SURE ?Y
01,FILES SCRATCHED,01,00
\end{verbatim}

Leider wird die Laufwerksangabe streng geprüft. Bei Angaben größer 1
wird ein \texttt{ILLEGAL QUANTITY ERROR} gemeldet. Dreifachlaufwerke
waren Commodore dann wohl doch nicht geheuer.

Immerhin sind bei der Angabe der Geräteadresse über ONU alle
zulässigen Werte möglich. Mit \texttt{CATALOG ONU 20} würde das
Inhaltsverzeichnis eines Laufwerks an Geräteadresse 20 angezeigt werden.
Werte größer als 31 oder kleiner als drei werden verweigert. 

BASIC prüft hier zwar falsch, da die größtmögliche Geräteadresse nicht 31
sondern nur 30 ist, aber wenigstens stellt das keine Einschränkung dar.

\subsection{Geräteadresse ändern}
Das Ändern der Geräteadresse Ihres Adapters erfolgt durch Senden
eines Befehls an den Kommandokanal mit Sekundäradresse 15. Eine einfache
Möglichkeit, mit BASIC-Bordmitteln Befehle zu senden, ist das Anhängen
des Befehls an den OPEN-Befehl.

\begin{verbatim}
OPEN 1,8,15,"XU=30" : CLOSE 1
\end{verbatim}

Dies sendet dem Adapter an Geräteadresse 8 über den Kommandokanal 15
den Befehl, fortan die neue Geräteadresse 30 zu verwenden.

Wenn Sie BASIC 4 haben, probieren Sie ein \texttt{CATALOG ONU 30}, ansonsten:
\begin{verbatim}
LOAD"$",30
LIST
\end{verbatim}

Für die Assembler- oder C-Programmierer kann es einfacher sein,
die Laufwerksangabe binär statt ASCII-dezimal zu übertragen. Mit diesem
zugegeberweise BASIC-Beispiel hierfür stellen wir die Geräteadresse
wieder von 30 zurück auf 8:

\begin{verbatim}
OPEN 1,30,15
PRINT#1,"XU=";CHR$(8)
CLOSE 1
\end{verbatim}

Nach dem Ausschalten verwendet der Adapter wieder die voreingestellte
Geräteadresse 8. Folgende Software-Versionen werden die Möglichkeit 
bieten, den eingestellten Wert dauerhaft auch über das Ausschalten hinweg
beizubehalten.

\clearpage %MANFF
\subsection{Laufwerken Inhalte zuweisen}
Bei klassischen Diskettenlaufwerken würden Sie eine Diskette 
einlegen. Stattdessen muss hier dem Adapter mitgeteilt werden, 
\textit{wo} wo die Daten liegen, die als virtuelle Diskette 
erscheinen sollen.

Weil der Zugriff auf eine Festplatte aber grundsätzlich anders abläuft
als beispielsweise der Zugriff auf einen FTP-Server im Internet, 
muss man dem Adapter auch noch mitteilen \textit{wie} er darauf
zugreifen soll.

Für jeden Speicherort (Dateisystem/FTP/HTTP) gibt es daher einen 
\textit{Provider.} Auf Deutsch bedeutet das etwa \textit{Lieferant}
oder \textit{Leistungserbringer}. 

Hier ein Beispiel mit vier zugewiesenen Laufwerken:

\begin{tabular}[c]{c l l}
\toprule
Laufwerk & Provider & Ort \\
\midrule
0 & Dateisystem & sample/ \\
1 & Dateisystem & tools/ \\
2 & FTP & ftp.zimmers.net \\
3 & HTTP & www.gutenberg.org/ebooks/ \\
\bottomrule
\end{tabular}

Beispiele für jeden Provider finden Sie im Kapitel \ 
\ref{assign} \glqq{}\nameref{assign}\grqq{} ab Seite
\pageref{assign}.

\clearpage
\section{Dateien und Verzeichnisse}
\subsection{Regeln für Datei- und Verzeichnisnamen}
Als maximale Länge eines Datei- oder Verzeichnisnamens sollten
16 Zeichen nicht überschritten werden, damit alte Programme nicht
durch unerwartet lange Dateinamen Schwierigkeiten bekommen.

Generell sollten Sie Sonderzeichen in Dateinamen unbedingt 
vermeiden. Absolut verboten sind:
\begin{itemize}
\item Doppelpunkte :
\item Schrägstriche /
\item umgekehrte Schrägstriche \textbackslash
\item der Klammeraffe @
\item das Paragraphenzeichen $\S$
\item der Stern *
\item das Fragezeichen ? 
\item das Dollar-Zeichen \$
\end{itemize}
Leerzeichen können dagegen problemlos verwendet werden.
Wenn Sie einen sichtbaren Trenner verwenden möchten, sollten
Sie das Minus-Zeichen '-' bevorzugen, da der Unterstrich '\_' 
auf einem CBM schwierig einzugeben sein kann.

\subsection{Wildcards -- Dateimasken}
Wildcards funktionieren im CBM-Umfeld anders, als Sie es von DOS,
Windows oder unixoiden Betriebssystemen gewöhnt sein werden, weshalb 
Sie sich wenigstens den nächsten Abschnitt \glqq \nameref{starwc}\grqq\ 
ansehen sollten.

Aber zunächst sei erst einmal sei erklärt, was Wildcards oder Dateimasken 
überhaupt sind: nämlich
eine Art Joker für Dateinamen, die das Arbeiten mit mehreren Dateien
auf einmal sehr erleichtern.

Das Fragezeichen '?' steht dabei für ein beliebiges vorhandenes einzelnes 
Zeichen, das '*' für mehrere beliebige Zeichen oder nichts.

Angenommen, Sie hätten eine Liste von Dateien:

\begin{verbatim}
BUDGET1977
BUDGET1978
BUDGET1979
BUDGET1979.Q1
BUDGET1979.Q2
BUDGET1979.Q3
\end{verbatim}

\texttt{@\$:BUDGET197?} würde dann BUDGET1977, BUDGET1978 und BUDGET 1979
anzeigen, nicht aber BUDGET1979.Q1, BUDGET1979.Q2 und BUDGET1979.Q3.

\texttt{@\$:BUDGET1979?} würde gar keine Datei anzeigen, da es keine
Datei gibt, deren Name nach \glqq BUDGET1979\grqq\ noch genau ein
weiteres Zeichen enthält.

\texttt{@\$:BUDGET1979*} würde alle vier Dateien BUDGET1979 und 
BUDGET1979.Q1, BUDGET1979.Q2 und BUDGET1979.Q3 anzeigen, da bei
einem Stern nicht notwendigerweise ein Zeichen folgen muss.

\subsubsection*{Arbeitsweise der '*'-Wildcard}
\label{starwc}
Im Unterschied zu anderen Betriebssystem erfolgt nach '*' keine
weitere Auswertung. Wenn Sie also versuchen würden, die Daten
der ersten Quartale durch \texttt{*.Q1} zu kopieren, werden Sie
feststellen, dass statt dessen \textit{alle} Dateien kopiert werden.
Probieren Sie es aus, indem Sie sich mit \texttt{@\$:*.Q1} die
etroffenen Dateien anzeigen lassen.

\subsection{Verzeichnisse}
Die Trennzeichen, die verwendet werden können, um einzelne 
Verzeichnisnamen innerhalb eines Pfades zu trennen, hängen vom
jeweiligen Provider und Betriebs"-system ab. 

Machen Sie es sich einfach, indem Sie ausschließlich
den einfachen Schrägstrich '/' verwenden, der immer und 
überall funktioniert.

Richtig: /Spiele/Strategie\\
Besser nicht: \textbackslash Spiele\textbackslash Strategie

Neben absoluten können auch relative Pfadangaben verwendet werden.
Was bedeutet das?

Nehmen wir an, Sie hätten Ihre Dateien 
in Unterverzeichnissen unter anderem wie folgt strukturiert:

\tikzstyle{every node}=[draw=black,thick,anchor=west]
\tikzstyle{selected}=[draw=red,fill=red!30]
\tikzstyle{optional}=[dashed,fill=gray!50]
\begin{tikzpicture}[%
  grow via three points={one child at (0.5,-0.7) and
    two children at (0.5,-0.7) and (0.5,-1.4)},
      edge from parent path={(\tikzparentnode.south) |- (\tikzchildnode.west)}]
\node {Hauptverzeichnis}
  child { node {Spiele} 
    child { node {Autorennen} }
    child { node {Ballern} }
    child { node [selected] {Strategie} }
  }
%  child [missing] {}
%  child [missing] {}
%  child [missing] {}
%  child { node {Texte} 
%    child { node {Arbeit} }
%    child { node {Briefe} }
%    child { node {Tagebuch} }
%  }
;
\end{tikzpicture}

Nehmen wir weiter an, sie befänden Sich im Unterverzeichnis Strategie
und wollten Sich nun in das Ballerspiel-Verzeichnis wechseln.

Absolute Pfadangaben beinhalten immer den kompletten Pfad vom
Hauptverzeichnis an. Mit \texttt{@CD:/SPIELE/BALLERN} würden Sie nun
das Verzeichnis wechseln.

Bei tiefer verschachtelten Verzeichnissen kann das eine Menge Tipp-Arbeit
bedeuten, weshalb sich relative Angaben anbieten. Zwei Punkte hintereinander
bezeichnen dabei immer das übergeordnete Verzeichnis, ohne dass
dieser Name ausdrücklich bekannt sein müsste. \texttt{@CD:../BALLERN}
bringt Sie dann direkt in das gewünschte Verzeichnis.

Mit \texttt{@CD:..} gelangen Sie immer eine Verzeichnisebene höher.

\texttt{@CD:/} bringt Sie in das Hauptverzeichnis.

\clearpage
\section{DOS-Befehle}
\subsection{Verwendete Notation}
\subsection*{Ersetzung durch tatsächlich passende Werte}
In spitzen Klammern und kursiver Schrift werden Werte angegeben, die 
Sie durch für Sie im Einzelfall tatsächlich
zutreffenden Werte ersetzen müssen. Beispiel:

\texttt{LOAD"}\textit{<Dateiname>}\texttt{",}\textit{<Geräteadresse>}

\texttt{LOAD"TEST",8}

\subsection*{Eine von mehreren Möglichkeiten wählen}
Soll aus mehreren Möglichkeiten genau eine Auswahl getroffen werden,
werden die Wahlmöglichkeiten in Klammern angegeben und durch einen
senkrechten Strich getrennt. Beispiel:

\mbox{\Big( Fisch \Big| \normalsize Fleisch \Big) \normalsize} 

Fleisch

\subsection*{Parameter angeben oder auslassen}
Wenn eine Angabe \textit{optional} ist, können Sie sie entweder
angeben oder weg lassen. Optionale Angaben stehen in eckigen Klammern.

\texttt{LOAD"} \Big[ Laufwerk-Nummer \texttt{:} \Big] \textit{<Dateiname>}
\texttt{",}\textit{<Geräteadresse>}

\texttt{LOAD"TEST",8} lädt die Datei TEST vom zuletzt 
verwendeten Laufwerk

\texttt{LOAD"2:TEST",8} lädt die Datei TEST von Laufwerk 2

\subsection*{Optionale Wiederholung}
Manche Befehle fordern mindestens einen Parameter, können aber beliebig
viele weitere\footnote{
	In der Praxis wird die Anzahl durch andere Umstände begrent, 
	wie etwa die Größe des Zwischenspeichers, der einen Befehl
	aufnehmen soll
} durch Kommata getrennte Parameter annehmen. Die optionale Wiederholung
wird dann durch ein Komma gefolgt von drei Punkten dargestellt:

\mbox{\textit{<Name>} \Big[ \texttt{,} \ldots \ \Big]}

\begin{verbatim}
PETER
PETER,PAUL,MARY
\end{verbatim}

\subsection{I -- Initialize -- Laufwerk initialisieren}

\texttt{I} \Big[ \textit{<Laufwerk-Nummer>} \Big]

Dieser Befehl ist nur aus Kompatibilitätsgründen enthalten.
Seine einzige Wirkung ist das Rücksetzen der Statusmeldung.

Beispiele:
\begin{verbatim}
I
I0
\end{verbatim}

\subsection{R -- Rename -- Datei oder Verzeichnis umbenennen}
Benennt eine Datei oder ein Verzeichnis um. Beachten Sie die Reihenfolge:
der neue Dateiname wird zuerst genannt.

\texttt{R}\Big[ \textit{<Laufwerk-Nummer>} \Big] \texttt{:}
\textit{<Neuer Name>} \texttt{=} \textit{<Alter Name>}

Beispiele:
\begin{verbatim}
@R:REZEPTE PIZZEN=REZEPT PIZZA
@R0:WUENSCHE2013.TXT=WUENSCHE2012.TXT
\end{verbatim}

\subsection{S -- Scratch -- Datei(en) löschen}
\texttt{S} \Big[ \textit{<Laufwerk-Nummer>} \Big] \texttt{:}
\textit{<Dateimaske>} \Big[ \texttt{,} \ldots \  \Big]

Löscht die angegebenen Datei(en). Mehrere Dateinamen bzw. Suchmuster
können durch Kommata getrennt angegeben werden.

Beispiele:

\texttt{@S:MIST} löscht die Datei MIST\\
\texttt{@S:STATISTIK198?.SEQ} löscht die Statistiken der Jahre 1980, 1981 
\ldots \ bis 1989 \\
\texttt{@S:MIST,UNSINN,ENTWURF,ENTWURF2} löscht alle genannten Dateien \\
\texttt{@S:*} löscht alle Dateien(!) des Unterverzeichnisses

\subsection{CD / CHDIR -- Change directory -- Unterverzeichnis wechseln}
\mbox{\Big( \texttt{CD} \Big| \texttt{CHDIR} \Big) 
\Big[ \textit{<Laufwerk-Nummer>} \Big]
\texttt{:} \textit{<Verzeichnis>}}

Wechselt in das angegebene Unterverzeichnis.

Beispiele:
\begin{verbatim}
@CD:SPIELE
@CD:..
@CD:TEXTE/HAUSAUFGABEN
\end{verbatim}

\subsection{MD / MKDIR -- Make directory}

\mbox{\Big( \texttt{MD} \Big| \texttt{MKDIR} \Big) 
\Big[ \textit{<Laufwerk-Nummer>} \Big]
\texttt{:} \textit{<Verzeichnis>}}

Erstellt das angegebene Unterverzeichnis. 

Bei Pfadangaben müssen
alle vorgenannten Verzeichnisse bereits existieren. Bei der
Angabe SOFTWARE/SPIELE/AUTORENNEN müssen also die Verzeichnisse
SOFTWARE/SPIELE bereits existieren, damit das Verzeichnis AUTORENNEN 
erstellt werden kann.

Beispiele:
\begin{verbatim}
@MD:SKIZZEN
@MD:TOPGAMES
\end{verbatim}



\subsection{RD / RMDIR -- Remove directory}

\mbox{\Big( \texttt{RD} \Big| \texttt{RMDIR} \Big) 
\Big[ \textit{<Laufwerk-Nummer>} \Big]
\texttt{:} \textit{<Verzeichnis>}}

Löscht das angegebene Unterverzeichnis, wenn es weder
Dateien noch Unterverzeichnisse enthält.

Beispiel:
\begin{verbatim}
@RD:EXPERIMENT2
\end{verbatim}


% ----------------------------------------------------------------
\label{xcommands}
% ----------------------------------------------------------------

\subsection{XU -- Change unit -- Geräteadresse ändern}

\texttt{XU=}\Big( \textit{<Geräteadresse in ASCII Ziffern>} \Big|
\textit{<Geräteadresse binär>} \Big)

Soll die Einstellung dauerhaft auch nach dem Ausschalten erhalten 
bleiben, muss die Konfiguraton durch den Befehl \texttt{XW} gesichert werden.

Beispiele:

\begin{verbatim}
@XU=9
OPEN 1,8,15,"XU=9" : CLOSE 1
OPEN 1,8,15 : PRINT#1,"XU=";CHR$(9) : CLOSE 1
\end{verbatim}

\subsection{XD - Change last drive - Letztes Laufwerk setzen}

\texttt{XD=}\Big( \textit{<Laufwerk-Nummer in ASCII Ziffern>} \Big|
\textit{<Laufwerk-Nummer binär>} \Big)

Das zuletzt verwendete Laufwerk wird als \glqq Standardlaufwerk\grqq\  für
Zugriffe verwendet, bei denen im Dateinamen durch vorangestelltes
\textit{<Laufwerk-Nummer>}: nicht ausdrücklich ein bestimmtes
Laufwerk angegeben wird.  
Zusätzlich kann dieses Laufwerk auch mit dem XD Befehl gesetzt werden.

Soll die Einstellung dauerhaft auch nach dem Ausschalten erhalten 
bleiben, muss die Konfiguraton durch den Befehl \texttt{XW} gesichert werden.

Beispiele:

\begin{verbatim}
@XD=0
OPEN 1,8,15,"XD=0" : CLOSE 1
OPEN 1,8,15 : PRINT#1,"XD=";CHR$(0) : CLOSE 1
\end{verbatim}

\subsection{XI -- Init -- Einschaltkonfiguration wieder herstellen}

\texttt{XI}

Wenn der Adapter über einen nicht flüchtigen Speicher für die
Konfiguration besitzt, wird die dort gespeicherte Konfiguration beim
Einschalten des Adapters automatisch wieder hergestellt. 
Ist das nicht möglich, werden Standard-Werte eingesetzt.
Mit \texttt{XI} kann diese Einschaltkonfiguration jederzeit wieder
hergestellt werden.

Standardkonfiguration, falls keine gespeicherte Konfiguration verfügbar:

\begin{tabular}[c]{l l l}
\toprule 
Einstellung & Befehl & Wert\\
\midrule
Geräteadresse		& XU= & 8 \\
Letztes Laufwerk	& XD= & 0 \\
\bottomrule
\end{tabular}

\clearpage %MANFF
\subsection{XW -- Write configuration -- Konfiguration schreiben}

\texttt{XW}

Speichert die aktuelle Konfiguration in einem nicht flüchtigem Speicher.
Beim nächsten Einschalten des Adapters werden die gespeicherten Werte
automatisch wieder hergestellt. Eine Übersicht der gespeicherten
Einstellungen ist in der Be\-schrei\-bung des \texttt{XI}-Befehls enthalten.

\subsection{XRESET - Adapter zurücksetzen}

\texttt{XRESET}

Dieser Befehl ist die Software-Entsprechung zum Druck auf die Reset-Taste
des Adapters. Alle nicht gespeicherten Daten und Einstellungen gehen dabei
verloren.

Die Daten des nichtflüchtigen Speichers bleiben erhalten. Die darin
gespeicherte Konfiguration kann sich jedoch von der zuletzt verwendeten
unterscheiden, wenn nicht zuvor ein \texttt{XW}-Befehl abgesetzt
wurde. 

Wenn noch Dateien zum Schreiben geöffnet sind, droht Datenverlust. 
Bei Dateisystemen, die auf dem Adapter gespeichert werden, wie etwa SD-Karten,
droht die Korrumpierung des gesamten Dateisystems, wenn zuvor
Schreibzugriffe erfolgt sind. Schließen Sie daher immer erst alle
geöffneten Dateien, bevor Sie diesen Befehl verwenden!

Wenn der Adapter über einen Bootloader zur Aktualisierung der Firmware
verfügt, wird dieser vor dem Starten der Firmware aktiv. \texttt{XRESET}
kann daher dazu verwendet werden, den Aktualisierungsvorgang anzustoßen.

\clearpage
\section{Konfiguration des Servers}
Der Server kann durch folgende Kommandozeilenparameter konfiguriert 
werden:

\texttt{fsser} \Big[ \textit{-<Option> \ldots \ \Big] }
\textit{<Laufzeit-Verzeichnis>}

\subsection{Laufzeit-Verzeichnis}
Beim Starten wechselt der Server in das angegebene Unterverzeichnis
\textit{<Laufzeit-Verzeichnis>}. Alle in den Optionen angegebenen
relative Pfadangaben beziehen sich dann auf dieses Laufzeitverzeichnis.

Beispiel: Ihre CBM-Software liegt in einem Ordner mit dem Pfad

\texttt{/home/benutzer-name/Software/Commodore/CBM/}

Sie möchten nun Laufwerk 0 und Laufwerk 1 belegen. 

Langform ohne relative Pfadangaben:

\begin{verbatim}
fsser 
-A0=fs:/home/benutzer-name/Software/Commodore/CBM/Visicalc
-A1=fs:/home/benutzer-name/Software/Commodore/CBM/Wordpro
-A2=fs:/home/benutzer-name/Software/Commodore/CBM/Assembler
-A3=fs:/home/benutzer-name/Software/Commodore/CBM/Spiele
-A4=fs:/home/benutzer-name/Software/Commodore/CBM/Spiele/Schach
.
\end{verbatim}

Kurzform mit Laufzeit-Verzeichnis und relativen Pfaden:

\begin{verbatim}
fsser 
-A0=fs:Visicalc
-A1=fs:Wordpro
-A2=fs:Assembler
-A3=fs:Spiele
-A4=fs:Spiele/Schach
/home/benutzer-name/Software/Commodore/CBM
\end{verbatim}

\subsection{Gerätedatei des Adapters}
\mbox{\Big[ \texttt{-d} \Big( \texttt{auto} \Big| 
\textit{<device>} \Big) \Big] }

Die Device-Option \texttt{-d} benennt die durch den
Adapter verwendete Gerätedatei, wie beispielsweise
\texttt{/dev/ttyUSB0}.

Wird diese Option nicht angegeben, wird 
\texttt{-d auto} verwendet.

Die automatische Erkennung \texttt{-d auto}
prüft, ob eine Reihe typischer Namen verfügbar sind, und wenn
genau einer dieser Möglichkeiten verfügbar ist, wird
diese verwendet. Findet der Server keine oder mehrere
Möglichkeiten wird er nicht starten und folgende Fehlermeldung
ausgeben:

\begin{verbatim}
Unable to decide which serial device it is.
Please pick one.
\end{verbatim}

Die Benennung unterscheidet sich je nach Betriebssystem.
Für den vom XS-1541 und petSD verwendeten Serial-over-USB Chip
FT232RL weist Linux 
fortlaufende Nummern beginnend bei \texttt{/dev/ttyUSB0} 
zu, während
\mbox{OS X} die im Chip gespei"-cherte Seriennummer verwendet. 
Diese Eigenart kann unter \mbox{OS X} dazu verwendet werden, die 
Seriennummer durch den Gerätenamen zu ersetzen.

\begin{tabular}[c]{l l l l}
\toprule 
Adapter & Betriebs- & Gerätedatei & Bemerkung\\
& system & & \\
\midrule
XS-1541 & Linux & /dev/ttyUSB0 & \\
XS-1541 & OS X & /dev/cu.usbserial-A10044Co & Standard-Seriennummer \\
petSD & Linux & /dev/ttyUSB0 & \\
petSD & OS X & /dev/cu.usbserial-petSD & Seriennummer bearbeitet \\
3.1541 & Linux & /dev/ttyAMA0 \\
\bottomrule
\end{tabular}

\subsection{Ausführlichkeit der Meldungen}
\mbox{\Big[ \texttt{-v} \Big]}

Wenn die Option \texttt{-v} \glqq{}verbose\grqq{} gegeben wird,
wird der Server eine Reihe weiterer ausführlicherer
Ausgaben machen, die bei der Fehlersuche behilflich sein können.

\subsection{Laufwerken Inhalte zuweisen}
\mbox{ \Big[ \texttt{-A}\textit{<Laufwerk-Nummer>} 
\texttt{:} \textit{<Provider>} \texttt{=}
\textit{<Parameter>} \Big] \Big[ \ldots \Big] }

Mit der \glqq{}assign\grqq{}-Option \texttt{-A} werden Laufwerke Inhalte
eines Providers zugewiesen. 

Einfach formuliert wird damit angegeben,
welche \glqq{}virtuelle Diskette\grqq{} in welches Laufwerk
eingeschoben wird. 

Provider können sowohl durch die XD-2031 Server Software,
die auf dem PC läuft, als auch durch den Adapter zur Verfügung
gestellt werden.

Neben dem Option \texttt{-A} beim
Serverstart können Zuweisungen auch vom CBM aus über das
Senden eines A-Befehls an den Kommandokanal \#15 erfolgen.  
Daher wird \textit{ASSIGN} in einem eigenem
Kapitel \ref{assign} \glqq{}\nameref{assign}\grqq{} 
auf Seite \pageref{assign} beschrieben.

\begin{comment}
TODO: is this feature broken?
Bei Providern, die der Server anbietet, ist zu beachten, 
dass aus Sicherheitsgründen zwar auf 
Unterverzeichnisse von \textit{<Parameter>} zugegriffen
werden kann, nicht jedoch auf Verzeichnisse in höherer Ebene.
\end{comment}

\subsection{Laufzeit-Konfiguration der Busse}
\mbox{\Big[ \texttt{-X} \textit{<Bus>} \texttt{:}
\textit{<Befehl>} \Big] \Big[ \ldots \Big] }

Mit Hilfe der X-Option können dem Adapter eine Reihe
X-Befehle übermittelt werden, die ab Seite \pageref{xcommands}
beschrieben werden.

Um beispielsweise die Geräteadresse für den IEC-Bus auf
9 und für den \mbox{IEEE-488} Bus auf 10 zu setzen:

\begin{verbatim}
-Xiec:U=9 -Xieee:U=10
\end{verbatim}

\clearpage
\section{Laufwerken Inhalte zuweisen}
\label{assign}

\subsection{Allgemeines über Provider}

\mbox{ \Big[ \texttt{A}\textit{<Laufwerk-Nummer>} 
\texttt{:} \textit{<Provider>} \texttt{=}
\textit{<Parameter>} \Big] \Big[ \ldots \Big] }

Mit der \glqq{}assign\grqq{}-Option \texttt{-A} werden Laufwerke Inhalte
eines Providers zugewiesen. Einfach formuliert wird damit angegeben,
welche \glqq{}virtuelle Diskette\grqq{} in welches Laufwerk
eingeschoben wird.

\begin{tabular}[c]{l l}
\toprule 
Provider & Zugriff auf \\
\midrule
fs & Dateisystem des PC \\
di & Disk-Images im Dateisystem des PC  \\
ftp & FTP-Server (über PC, nur lesend) \\
http & Webserver (über PC, nur lesend) \\
\bottomrule
\end{tabular}

\subsection{fs: Zugriff auf das Dateisystem des PC}
\mbox{ \texttt{A}\textit{<Laufwerk-Nummer>} 
\texttt{:} \textit{fs} \texttt{=}
\textit{<Pfad-im-PC-Dateisystem>} }

Der filesystem-Provider bietet Schreiblesezugriff auf das
angegebene Verzeichnis. Bei relativen Pfadangaben beziehen sich diese
auf das Laufzeit-Verzeichnis des Servers, bei absoluten Pfadangaben
spielt das Laufzeit-Verzeichnis natürlich keine Rolle.

In diesem Beispiel wird der absolute Pfad /usr/local/xd2031/tools
dem Laufwerk 0 zugewiesen:

\begin{verbatim}
fsser -A0:fs=/usr/local/xd2031/tools .
\end{verbatim}

Das Laufzeit-Verzeichnis des Servers wurde der Einfachkeit halber
auf das aktuelle Verzeichnis '.' gesetzt.

Ein Beispiel mit relativen Pfaden:

\begin{verbatim}
fsser -A0:fs=office/text ~/Software/Commodore/CBM
\end{verbatim}

Pfadangaben sind relativ zum Laufzeit-Verzeichnis, 
so ergibt sich der Gesamt"-pfad  
\texttt{\~{}/Software/Commodore/CBM/office/text}.
Dieses Verzeichnis steht nun als Laufwerk 0 
für Schreib- und Lesevorgänge zur Verfügung.

\subsection{di: Zugriff auf Disk-Images im Dateisystem des PC}
\mbox{ \texttt{A}\textit{<Laufwerk-Nummer>} 
\texttt{:} \textit{di} \texttt{=}
\textit{<Disk-Image-im-PC-Dateisystem>} }

Mit dem Disk-Image-Provider können Disk-Images einem
Laufwerk zugewiesen werden. Ein Disk-Image enthält einen 1:1 Abzug
einer Diskette eines bestimmten Diskettenlaufwerks, inklusive
aller Daten des Dateisystems.

Die Erkennung des Typ des Disk-Images erfolgt unabhängig vom 
Dateinamen aufgrund der Dateigröße. 

Es gibt verschiedene Varianten, die sich durch die Anzahl
der hinterlegten Spuren (35 / 40) oder eine eventuell 
angehängte 
Fehlertabelle\footnote{Status-Code des Diskettencontrollers für 
jeden Block des Abbildes} 
unterscheiden können. Auch Kombinationen daraus sind möglich.

Varianten werden derzeit noch nicht unterstützt, lediglich die
Standard-Images.

Folgende Disk-Images werden unterstützt:

\begin{tabular}[c]{l l r}
\toprule 
Typ & Original-Laufwerke & Größe in Bytes \\
\midrule
D64 & 1541, 1571, 2031, 4031, 4040 & \numprint{174848} \\
D80 & 8050, 8250 & \numprint{533248} \\
D81 & 1581 & \numprint{819200} \\
D82 & 8250 & \numprint{1066496} \\
\bottomrule
\end{tabular}

Die Verwendung von Disk-Images bietet sich dort an,
wo die Erhaltung von Sonderzeichen im Dateinamen oder die
Anordnung innerhalb des Inhaltsverzeichnis wichtig ist.

Beispiel: die \glqq{}virtuelle Diskette\grqq{}
\texttt{visicalc.d82} wird in Laufwerk 0 eingelegt:

\begin{verbatim}
fsser -A0:di=office/calc/visicalc.d82 ~/Software/Commodore/CBM
\end{verbatim}

\clearpage
\section{Anhang}
\subsection{Verwendete Lizenzen}
\subsubsection{Benutzerhandbuch}
Dieses Handbuch ist unter der
\textit{Creative Commons CC BY-SA 3.0} lizensiert.

\copyright{} 2013 Nils Eilers

Eine Beschreibung und den rechtsverbindlichen Lizenzvertrag
können Sie unter \url{http://creativecommons.org/licenses/by-sa/3.0/de/}
einsehen.

\includegraphics[scale=1]{by-sa.pdf}

\subsubsection{XD-2031}
Das Programm \glqq{}XD-2031\grqq{} ist unter der
\textit{GNU General Public License, Version 2} lizensiert.

\copyright{} 2012-2013 André Fachat, Ingo Korb, Thomas Winkler,
Nils Eilers und andere.

Der rechtsverbindliche Lizenzvertrag ist im folgenden angehängt.
Bitte beachten Sie, dass der PC Server auch unter späteren
Versionen der GPL verfügbar ist, die ATmega Firmware dagegen
\textbf{nur unter GPL Version 2}. Die tatsächlich anwendbare
Lizenz ist im Quelltext der jeweiligen Datei enthalten.

\input{gpl-2.0.tex}

% Erzeugt falschen Link im PDF
% \addcontentsline{toc}{subsection}{Stichwortverzeichnis}

% \renewcommand{\indexname}{Stichwortverzeichnis}
\printindex

\end{document}




