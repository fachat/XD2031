\documentclass[10pt,a4paper]{scrartcl}		% Ersatz für Klasse "article" 
\parskip 12pt
\parindent 0pt
\usepackage{makeidx}					% Stichwortverzeichnis
\makeindex
\makeglossary
%\usepackage[ngerman]{babel}				% Umlaute "a "s etc.
\usepackage[utf8]{inputenc}				% Umlaute in Quelldatei UTF8-kodiert
\usepackage[T1]{fontenc}
\usepackage{footnpag}					% Nummerierung auf jeder Seite neu beginnen
\usepackage[hyphens]{url}
\usepackage{listings}
\lstloadlanguages{[Visual]Basic}
\lstset{language=C, basicstyle=\ttfamily\small, commentstyle=\itshape }
\usepackage{numprint}
\usepackage{booktabs}
\usepackage{amssymb}					% $\checkmark$
\usepackage[colorlinks,
pdfpagelabels,
pdfstartview = FitH,
bookmarksopen = true,
bookmarksnumbered = true,
linkcolor = black,
plainpages = false,
hypertexnames = false,
citecolor = black] {hyperref}
\usepackage{eurosym}
\usepackage{hyperref}
\usepackage{tikz}
\usetikzlibrary{trees}
\usepackage{ifthen}

% **************   SERIF   **************
%\usepackage{mathptmx} % Times Roman
%usepackage{fouriernc} % New Century Schoolbook
\usepackage{tgschola} % TEX Gyre Schola (based on New Century Schoolbook)
%\usepackage[scaled]{DejaVuSerif}

% **************   SANS SERIF   **************
%\usepackage[scaled=.90] {helvet}
\usepackage{tgadventor} % TEX Gyre Adventor (based on URW Gothic, an Avantgarde clone)
%\usepackage{tgheros} % TEX Gyre Heros (based on URW Nimbus Sans)
%\usepackage{arev} % Arev is a version of Bitstream Vera Sans, disigned for slide presentations

% **************   MONO   **************
%\usepackage{luximono}
\usepackage{beramono}

\newcommand{\beginsf}[1]{#1\begin{addmargin}[0.5cm]{0cm}}
\newcommand{\sfend}[1]{\end{addmargin}}

% -------------------------------------------------------------------------
% Variablen
%
\newcommand{\fwbinaries}{http://FIXME-fwbinaries.de}
\newcommand{\osxbinaries}{http://FIXME-osxbinaries.de}
% -------------------------------------------------------------------------

% -------------------------------------------------------------------------
% Global variables for LaTeX documents
% -------------------------------------------------------------------------

% Internet links
\newcommand{\fwbinaries}{\url{http://xd2031.petsd.net/firmware.php}}
\newcommand{\osxbinaries}{\url{http://xd2031.petsd.net/serverosx.php}}
\newcommand{\srcdownloads}{\url{https://github.com/fachat/XD2031/tags}}


\begin{document}

\newcommand{\screenshot}[2][scale=0.5]{ 
	\begin{center} 
	\includegraphics [ #1 ] {images/#2} 
	\end{center}  
}

\newcommand{\clickhere}[1]{\textit{#1}}
\newcommand{\typethis}[1]{\textbf{#1}}

\title{How to compile XD-2031}
\author{}
\maketitle
\clearpage

\tableofcontents
\clearpage

\section{About}
This document describes how to compile the XD-2031 software. 
It should give you some hints if you haven't worked with 
GNU make / GCC / Eclipse
before. If you're lucky enough to have this practice, you won't find
much if any news here.

Instructions contained herein cover both compiling the firmware and the
server software under Linux and OS X. Sorry, there isn't a Windows port 
at the time.

You can keep your favorite editor and compile at the command line with 
GNU make, but the usage of the integrated development environment 
\textit{Eclipse} will be explained also.

Both, firmware and server, are written in plain C.

For any hints, how this software works,
please have a look at the \textit{/docsrc} 
and \textit{/doc} directories and read \textit{firmware overview.odp} 
(Libre Office document) and \textit{wireformat.txt}.

You also won't find detailed explanations on how to contribute patches although
they are welcome. Cloning the git repository and attempting a 
\textit{pull request} would be the best way. Simply drop a mail in that case.

Good luck and may the Force be with you...

\section{Prerequisites}

\subsection{Linux}
Freedom causes variety, by consequence the names of the required packages 
and the commands to install these packages vary from distribution 
to distribution.
You are of course welcome to contribute commands for your preferred distribution. 

\subsubsection{Debian / Ubuntu style Linux (including Mint)}
For an easy, complete installation (besides Eclipse) 
\href{apt://build-essential,gcc-avr,avr-libc,avrdude,avrdude-doc,libcurl4-gnutls-dev,git,git-gui,doxygen,graphviz,texlive,texlive-latex-extra,texlive-fonts-extra,texlive-lang-german,hunspell,hunspell-en-us,hunspell-de-de}{click this apturl-link}.
While the installation process is still running, you can immediately 
continue with the 
\nameref{eclipseinst} on page \pageref{eclipseinst}.

If you prefer a manual installation, open a terminal and copy the following line into it:

\begin{verbatim}
sudo apt-get install build-essential gcc-avr avr-libc 
avrdude avrdude-doc libcurl4-gnutls-dev
\end{verbatim}

Make sure to copy this into a single command line, then hit ENTER 
and type in your password to install the packages. 
This may take a while...  

If you get a message, that you are not in the 
sudoers list, you may gain root rights with \texttt{su <your-root-password>} 
and copy the rest of the line after sudo.

The usage of git is strongly recommended, but not necessary if you want to
compile only a specific release version of XD-2031. If git isn't installed
at your system by default, you may add it with:

\begin{verbatim}
sudo apt-get install git git-gui
\end{verbatim}

The following packages are optional, but may be useful: 

Doxygen is a
documentation generator. It is a great help to get an overview over all
functions, data types etc. 

\begin{verbatim}
sudo apt-get install doxygen graphviz
\end{verbatim}

This manual was made with \LaTeX{}. Be warned: this is a rather huge
download and may take a while to install, so go and grab a coffee...

\begin{verbatim}
sudo apt-get install texlive texlive-latex-extra 
texlive-fonts-extra texlive-lang-german hunspell 
hunspell-en-us hunspell-de-de
\end{verbatim}

\subsubsection{Installation of the Eclipse IDE}
\label{eclipseinst}
Open the Eclipse download page and download
either the 32 or 64 bits version of
"Eclipse IDE for C/C++ Developers":

\url{http://www.eclipse.org/downloads/}

Extract and move the files with the following commands:

\begin{verbatim}
tar zxvf eclipse-cpp-juno-SR2-linux-gtk-x86_64.tar.gz 
sudo mv eclipse /opt
sudo ln -s /opt/eclipse/eclipse /usr/bin/eclipse
\end{verbatim}

It is possible now to start Eclipse by typing \texttt{eclipse} at
the command line, but you may want to add an appropriate entry
in your main menu. If you're not sure how to do this,
\href{apt://alacarte}{alacarte} is a program that might help you.

\subsection{OS X}
If you haven't already done before, start with installing XCode. 
This huge software environment including a larger number of applications
not needed for compiling XD-2031 is virtually the only 
way to get a proper GCC for the Mac.

If you still are on Snow Leopard (or even before that), XCode is
included in your installation disks. You can not install XCode using
the App Store since that version requires Lion at least.

If you are on Lion, Mountain Lion or whatever cat to come next (are
there still any cats left?), install XCode via the App Store:

\url{https://itunes.apple.com/us/app/xcode/id497799835?mt=12&uo=4}

Next, install the AVR tool-chain "CrossPack for AVR". Grab and install it from: 

\url{http://www.obdev.at/products/crosspack}

Starting with Mountain Lion, you may have to (at least temporarily) disable
Gatekeeper to be able to install software from beyond the App Store. Instructions can be found at \url{https://answers.uchicago.edu/page.php?id=25481} or
on YouTube at \url{http://youtu.be/H5DZXA4uxeQ}.

The usage of git is strongly recommended, but not necessary if you want to
compile only a specific release version of XD-2031. Don't think further, 
go for it: 

\url{http://git-scm.com/download}

If you'd like to use the Eclipse IDE, visit their website  
and make sure to download the "Eclipse IDE for C/C++ Developers". 
Install it as usual.

\url{http://www.eclipse.org/downloads} 

\subsection{List of software versions}
Everything changes. This manual was made for the software versions
listed below. Other versions may also work but have not necessarily been 
tested.

\begin{tabular}[c]{l l l}
\toprule
Package		& Description	& Version \\
\midrule
GNU make 	& build tool			& 3.81 \\
GCC		& GNU C compiler		& 4.7.2 \\
cURL		& cmdline tool for transferring files & 7.27.0 \\
AVR-GCC 	& GNU C compiler for AVR	& 4.7.0 \\
AVR Libc	& C library for AVR		& 1.8.0 \\
AVRDUDE		& AVR Downloader/UploaDEr	& 5.11.1 \\
git				& Version control system 	& 1.7.11.2 \\
Doxygen				& Documentation generator 	& 1.8.1.2 \\
Eclipse IDE for C/C++ 		& IDE				& Juno SR 2 \\
\LaTeX{}			& Typesetting program 		& TeX Live 2012 \\
\bottomrule
\end{tabular}

\section{Download Sources}

XD-2031 is still under heavy construction, so you might want 
to get the latest developer sources with git. However, each commit may
introduce new bugs, so you may consider the use of a release version or
a snapshot taken a few days ago as well.

\subsection{Releases / Snapshots}
Release versions and git snapshots are available from the URL given below. 
Download and unzip your desired source in a directory of your choice.

\url{http://xd2031.petsd.net}


\subsection{Latest developer sources}

To clone the repository, run the following command. It will create the directory 
\textit{XD2031} inside the current directory, download the latest source and place it there.
This step is required only once. If you later decide to move this directory, you can do so.
It is not necessary to delete and clone it again somewhere else.

\begin{verbatim}
git clone https://github.com/fachat/XD2031
\end{verbatim}

This tutorial can not and doesn't want to be an introduction into git,
and using git is not necessary for compiling and playing around if you
do not intend sharing your results. On the other hand, a version control system is
a great help and if you haven't worked with a VCS before or are new to git, it is definitely 
worth doing further reading. 
Start here: \url{http://git-scm.com/doc}

\section{Command line tools}
\subsection{Firmware}
\subsubsection{Declare default device}
Compilation is controlled by the build control tool 'GNU make'. If you haven't worked with Makefiles before,
this section gives you some hints on usage.

In most cases you'll want to compile for your specific device only. 
You do so by declaring exactly one device as your default device inside
the Makefile. Open the file \textit{Makefile} with your favorite editor.
You will find:

\clearpage
\begin{verbatim}
# Uncomment one of the following lines to select your device:
DEVICE=xs1541
#DEVICE=petSD
\end{verbatim}

Comments start with a '\#', so all device definitions starting with a '\#'
are in fact only comments and thus disabled. If you want to change the
default device, delete the '\#' before its line and don't forget to add
another '\#' at the line, that has been the default device until then.
If you'd like to compile for a petSD, your Makefile should therefore look
like this (notice how the '\#' moved from petSD to xs1541):

\begin{verbatim}
# Uncomment one of the following lines to select your device:
#DEVICE=xs1541
DEVICE=petSD
\end{verbatim}

If you edit the Makefile and save your changes, make sure your editor does 
not replace tabs with spaces or the Makefile won't work any longer.

\subsubsection{Compile}

Once you defined your default device, compilation is made by typing simply make without parameters. The binaries are placed in \textit{XD2031/firmware/bin}.

\begin{verbatim}
cd XD2031/firmware
make
\end{verbatim}

\subsubsection{Compile for all devices}
You can compile for all devices with a single command. 
GNU Make will run all necessary steps and compile each target one by one. The binaries are then placed in XD2031/firmware/bin.

\begin{verbatim}
cd XD2031/firmware
make zoo
\end{verbatim}

\subsubsection{Upload firmware}
Edit \textit{firmware/Makefile} suitable for your device and programmer:

\begin{verbatim}
# This defines the serial-over-USB port to use when loading the
# firmware into the device with "make load"
# For AVR (AtMega) needs avrdude - see avr/Makefile for details
SERIAL=/dev/ttyUSB0

# If you do ISP programming with "make flash", select your programmer/port:
DUDE_PROGRAMMER = avrispmkii
# DUDE_PROGRAMMER = stk200

DUDE_PORT = usb
# DUDE_PORT = /dev/parport0
# DUDE_PORT = lpt1

# If you didn't find your programmer, have a took at the list given from:
# avrdude -c ?
\end{verbatim}

To upload the firmware on a XS-1541 via bootloader / USB, enter at your
console without pressing ENTER:

\begin{verbatim}
make load
\end{verbatim}

Reset the XS-1541 and hit ENTER within the next three seconds.

When using an ISP programmer (like an AVR ISP mkII), simply type:

\begin{verbatim}
make flash
\end{verbatim}

\subsection{Server}
\subsubsection{Compile}
To compile the Server, change into the directory \textit{XD2031/pcserver}
and run the Makefile just the way you did to compile the firmware:

\begin{verbatim}
cd XD2031/pcserver
make
\end{verbatim}

Your binary, let's say your program \textit{fsser},  will be placed 
in the current directory \textit{XD2031/pcserver}.

\subsubsection{Install}
If you'd like to run the Server with its default configuration
(without parameters), installation is required to move
some files to places where the Server expects them. To permanently install
the Server, run:

\begin{verbatim}
sudo make install
\end{verbatim}

To check respectively edit what goes where, have a look at the head of \textit{Makefile}
inside the pcserver directory. Watch out for:

\begin{verbatim}
PREFIX=/usr/local
BINDIR=bin
DOCDIR=xd2031/doc
SAMPLEDIR=xd2031/sample
TOOLSDIR=xd2031/tools
\end{verbatim}

The full path for binaries, documents, samples and tools is generated by
appending the specific directory path to PREFIX. As an example:

\texttt{PREFIX=/usr/local \\
DOCDIR=xd2031/doc \\
$\to$ /usr/local/xd2031/doc}

Please be aware of the difference between the two document directories:

\textit{XD2031/doc} should contain documents relevant while
using XD-2031. These files are copied by the installer during installation.

The opposite is true for \textit{XD2031/docsrc}: this directory should
contain files that are of interest during development only but not
for a regular user, so the installer ignores those files.

\subsubsection{Uninstall}
To uninstall the Server, run:

\begin{verbatim}
sudo make uninstall
\end{verbatim}

This will remove not only the Server but also all its directories that may
contain user-generated files, such as \textit{/usr/local/xd2031/samples}.
So please be careful and check first, if you want to save some of your
files first, before uninstalling.

On OS X, the auto-generated uninstall script from xd2031.dmg behaves
slightly different: it will remove the Server and all the files it came with. 
It will \textit{not} delete any files you created by yourself, e.g. by
SAVEing to a sample directory.

\subsubsection{Privileged Server Start (Linux)}
\label{rights}
On OS X, the device (XS-1541 / petSD) is always accessible with
user rights. This is not the case for Linux.

Some Linux systems are very restrictive and doesn't allow a program running
with user rights to communicate with an attached device, e.g. a XS-1541
showing up as /dev/ttyUSB0. Enabling this can be challenging, so by default
the Server starts with root rights to ensure access to the device.
Once opened, it drops his privileges and continues running with user rights
to make sure not to harm anything.

That's the reason, why you will be asked for your password when compiling.
Sudo is required to enable the Server to start with root rights.

If you want to configure the server to always use user rights only, make sure
it can access the device. As a hint:

\begin{verbatim}
ls -l /dev/ttyUSB0
\end{verbatim}

to get the group the device was assigned too. This could be \textit{dialout}.
Then add your user to this group. In most cases, you should be able to
access the device now with user rights only.

To compile for "user rights only", edit 
\textit{pcserver/Makefile} and change PRIVILEGED from 'y' to 'n'.
Your resulting section of the Makefile should look like this:

\begin{verbatim}
# Open the serial device with root or user rights
PRIVILEGED=n
\end{verbatim}

Another possibility without editing the Makefile is passing this
case-sensitive parameter when calling make:

\begin{verbatim}
make PRIVILEGED=n
\end{verbatim}

\subsubsection{Generate dmg for binary distribution (OS X)}
If you want to use your self-compiled Server on more than a single Mac,
you can generate a nice xd2031.dmg file:

\begin{verbatim}
make dmg
\end{verbatim}

\screenshot{xd2031_dmg}

\clearpage
\subsubsection{Doxygen}
Doxygen is a documentation generator. Its a great help to get
an overview over all functions, data types etc. And last but not least,
it can even draw some nice figures:
\screenshot[scale=0.4]{serial_graph}

To generate the documentation for the firmware, change into the firmware
directory and invoke make. 

\begin{verbatim}
cd XD2031/firmware
make doc
\end{verbatim}

Open \texttt{XD2031/docsrc/doxygen/firmware/html/index.html}
with a browser (Firefox, Safari...) to view the generated documentation.

Generating the documentation for the Server is just the same.

\begin{verbatim}
cd XD2031/pcserver
make doc
\end{verbatim}

Open \texttt{XD2031/docsrc/doxygen/pcserver/html/index.html}.


\clearpage
\section{Configure Eclipse}
\subsection{Workspace}
Click \clickhere{File $\to$ Switch workspace $\to$ Other...}

Select the topmost XD2031 directory as your workspace and click OK.

\screenshot{eclipse_workspace}

\clearpage
\subsection{Install AVR plugin}
Click \clickhere{Help $\to$ Install New Software...}

Fill the field \clickhere{Work with} with \typethis{http://avr-eclipse.sourceforge.net/updatesite}

Click \clickhere{Add...}.

\screenshot{eclipse_avr2}

\clearpage
Give a name, e.g. \typethis{AVR-Eclipse}, then click OK.
\screenshot{eclipse_avr3}

Select the newest version. Make sure to checkmark it. Then click \clickhere{Next}
a couple of times to install the plugin.
\screenshot[scale=.4]{eclipse_avr5}

\clearpage
\subsection{Editor preferences}
If you're on OS X, click at \clickhere{Eclipse $\to$ Preferences...} 
On Linux or Windows, this should be under
\clickhere{File $\to$ Preferences...} 

Expand \clickhere{General $\to$ Editors $\to$ Text Editors} at the left side of the window.

Set tab size to \typethis{8}, then click \clickhere{Apply}.

\screenshot{eclipse_tab}

\clearpage
Expand \clickhere{C/C++ $\to$ Editor $\to$ Save Actions} at the left side of the window.

Make sure to have removing whitespaces \clickhere{In edited lines} selected, or completely disable
this function. Do not select \clickhere{In all lines} or you'll end up with a lot of whitespace only commits when contributing with git.

Click \clickhere{Apply}.  You may check other options now and make them suitable for your needs, then click OK.
\screenshot{eclipse_whitespace}

\subsection{Firmware Project}
Click \clickhere{File $\to$ New $\to$ Makefile Project with Existing Code...}

Click \clickhere{Browse...} and select the firmware directory of your previous downloaded \mbox{XD-2031} source
tree. The \clickhere{Project Name} will automatically get filled with \clickhere{firmware}, what is
just fine.

Select the \clickhere{AVR-GCC Toolchain}, then click \clickhere{Finish}.

\screenshot{eclipse_firmware_project}

In order to be able to upload the firmware with a single mouse-click, we have to switch 
our new project into an AVR project. There's still some magic required to do this...

Click \clickhere{Window $\to$ Show View $\to$ Project Explorer}. Double click at \clickhere{firmware} to expand the files list. You should see a number of source files and directories now:
\screenshot[scale=.3]{eclipse_project_explorer}


Click at the triangle icon to expand the view menu. Click \clickhere{Customize View..}. 

\screenshot{eclipse_view_menu}

\clearpage
Remove the checkmark at \clickhere{.* resources} to show hidden files. Click OK.
\screenshot{eclipse_view_menu2}

At the end of the file list, some hidden dot-files are visible now. You may have to scroll down to
see them. Double-click at \clickhere{.project} to open the file.

Add the following line to the <natures>-Block:

\texttt{<nature>de.innot.avreclipse.core.avrnature</nature>}

Save the file by clicking \clickhere{File $\to$ Save}.

\screenshot[scale=.4]{eclipse_avr_nature}

Click at the AVR Upload Icon. You should get a message \clickhere{No programmer has been set for the
project} if you successfully switched your project into an AVR project.
Click \clickhere{File $\to$ Restart} then to restart Eclipse. You should get the same view again.
\screenshot{eclipse_no_programmer}

\subsection{AVRDUDE}

Click \clickhere{Project $\to$ Properties} at the menu, and expand \clickhere{AVR $\to$ AVRDude} at the left
side of the \clickhere{Properties for firmware} window.

Click \clickhere{New...} in the section \clickhere{Programmer configuration}.

The following configuration fits for a AVR ISP mkII. If you're still in need of a programmer,
don't waste time and money with toys, go and buy a tool like this. 

Name the configuration, e.g. \typethis{AVR ISP mkII}. Select your appropriate programmer, 
\typethis{Atmel AVR ISP mkII} here. Enter \typethis{usb} at the field \clickhere{Override default port (-P)}.

\screenshot[scale=0.3]{eclipse_avrdude}

Click OK to get back to the window \clickhere{Properties for firmware}. 

Click \clickhere{AVR $\to$ Target Hardware} at the left side of the \clickhere{Properties for firmware} window.

Select the MCU Clock Frequency appropriate for your device:

\begin{tabular}[c]{l l l}
\toprule
Device		& MCU Clock Frequency \\
\midrule
XS-1541 & 14745600 \\
petSD	& 18432000 \\
\bottomrule
\end{tabular}

At \clickhere{MCU Type} above, click \clickhere{Load from MCU} to auto detect it. This is also a test
to verify the communication with the programmer. You should get one of those ATmega644-types for 
an XS-1541 and ATmega1284P for a petSD.
\screenshot[scale=0.4]{eclipse_avrdude2}

Click OK to close the properties window.

At the \clickhere{Project Explorer}, double click at \clickhere{Makefile} to open the file.
Uncomment one of the DEVICE definitions to select your device. Make sure to comment a previous
uncommented device, since only a single device is allowed at once.

Click \clickhere{File $\to$ Save} to save the edited Makefile.
\screenshot[scale=.4]{eclipse_Makefile}

Now its time to compile for the first time: click \clickhere{Project $\to$ Build All}. The compiler output is shown in the Console window. Now that we have
binaries in the directory \textit{bin}, we can configure the 
AVRDude Plugin to flash those.

\clearpage

Click \clickhere{Project $\to$ Properties} and expand 
\clickhere{AVR $\to$ AVRDude} at the left side of the 
\clickhere{Properties for firmware} window.

Open the section \clickhere{Flash / EEPROM}. 
At the section \clickhere{Upload Flash Memory Image}, select the radio-box
\clickhere{from Flash memory image file}. Click at
\clickhere{Workspace...}. A file selection window pops up then.
Expand the directory firmware to see all files, go to the bin
directory and choose the hex-file appropriate for your device.

\screenshot[scale=0.4]{eclipse_avrdude3}

Click OK to close the properties window.

Congrats, you should be able to upload the firmware now simply by clicking
the AVR Upload Icon. If you prefer using the menu bar, click
\clickhere{AVR $\to$ Upload Project to Target Device}.

\subsection{Server Project}
Configuring the Server Project is much like the firmware project. 
Click \clickhere{File $\to$ New $\to$ Makefile Project with Existing Code...}

Click \clickhere{Browse...} and select the directory \clickhere{pcserver} of your previous 
downloaded \mbox{XD-2031} source tree. The \clickhere{Project Name} will automatically get filled 
with \clickhere{pcserver}, what is just fine.

Select the toolchain appropriate for compiling for your computer, \clickhere{MacOSX GCC} here, 
then click \clickhere{Finish}.

\screenshot{eclipse_pcserver_project}

\clearpage
Now click \clickhere{Project $\to$ Clean...} 

Select 
\clickhere{Clean projects below}. Checkmark only \clickhere{pcserver}.
Checkmark \clickhere{Start a build immediately} and select
\clickhere{Build only the selected projects}. Then click OK to 
compile the Server.

\screenshot{eclipse_clean}

Have a look at the Console Log and check, if the compilation was
successful. If you read \textit{fsser opens the serial device...}
everything went fine and you can skip the rest of this chapter.

If you configured the Makefile to compile the Server starting
with root rights, you may find:

\begin{verbatim}
sudo chown root fsser
sudo: no tty present and no askpass program specified
\end{verbatim}

You should have read the chapter \nameref{rights} at page \pageref{rights}
first to understand why your password is required at compile time.

\clearpage
There are two workarounds:

First you may edit \textit{pcserver/Makefile} and replace sudo with 
gksudo (GNOME, Xfce, LXDE) or
kdesudo (KDE). On the downside, you may have to enter your
password twice at each compilation.

Another  possibility is allowing the execution of \texttt{chmod}
and \texttt{chown} with root rights without the need of entering
the password. 

\textbf{If you disable the sudo password for your account, you 
will seriously compromise the security of your computer. Anyone sitting 
at your unattended, logged in account will have complete Root access, 
and remote exploits become much easier for malicious crackers.}

You have been warned...

Start with getting the absolute paths of chown and chmod by entering
the commands:

\begin{verbatim}
whereis chown
whereis chmod
\end{verbatim}

This gives \texttt{/usr/sbin/chown} and \texttt{/bin/chmod} here.

Add the following lines with \texttt{sudo visudo} at the end
of your /etc/sudoers file:

\begin{verbatim}
<your-user-name> ALL = NOPASSWD: /usr/sbin/chown
<your-user-name> ALL = NOPASSWD: /bin/chmod
\end{verbatim}

Replace <your-user-name> with (you guessed it) your user name and
omit the <>. And don't forget to give the paths to chown and chmod
suitable for your system, that's what you asked whereis for.

\end{document}
